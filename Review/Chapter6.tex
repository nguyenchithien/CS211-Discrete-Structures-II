\documentclass[a4paper,12pt]{book}
\usepackage[utf8]{inputenc}
\title{}
\author{Rachel Morris}
\date{\today}

\usepackage{rachwidgets}
\usepackage{fancyhdr}
\usepackage{lastpage}
\usepackage{dirtree}
\usepackage{boxedminipage}

\setcounter{chapter}{6}
\setcounter{section}{0}
\newcommand{\laChapter}{Chapter 6 Review\ }

\newcommand{\laClass}{CS 211\ }
\newcommand{\laSemester}{Fall 2017\ }
\newcounter{question}

\pagestyle{fancy}
\fancyhf{}
\lhead{CS 211}
\chead{Fall 2017}
\rhead{\laChapter Review}
\rfoot{\thepage\ of \pageref{LastPage}}
\lfoot{\scriptsize Compiled by Rachel Morris, last updated \today}

\renewcommand{\headrulewidth}{2pt}
\renewcommand{\footrulewidth}{1pt}

\begin{document}

    %------------------------------------------------------------------%
    \chapter*{\laChapter}

    %------------------------------------------------------------------%
        \section*{Formulas, definitions, and theorems}

        \paragraph{Experiment:}
            An experiment can be anything that can have multiple outcomes.
            Usually in this chapter it is rolling a die, drawing a card,
            or flipping a coin.
        
        \paragraph{Sample space $S$:}
            The sample space $S$ is a \textbf{set} of all possible
            \textit{equally-likely} outcomes of the experiment.
            
            For example, the sample space $S$ of rolling a die is
            \{1, 2, 3, 4, 5, 6\}, and the length $n(S)$ is 6.
        
        \paragraph{Event:}
            When performing an experiment, we will ask how likely some
            \textit{event} is to occur. If the event occurs, we will
            call this a \textit{success}.
            
            For example, if our event $E$ is rolling a die and getting
            an even number, then $E$ is \{2, 4, 6\}. The length of
            this, $n(E)$, is 3.
        
        \paragraph{Probability of an event $E$ taking place:}
            The probability that $E$ occurs is written as $Prob(E)$,
            and can be calculated as $\frac{n(E)}{n(S)}$.
            
            For example, the probability of rolling a die and getting
            an even number is $\frac{3}{6}$, or $\frac{1}{2}$.

        \paragraph{The complement $\bar{E}$ of an event $E$:}
            For an experiment, an event $E$ can either happen or not
            happen. The sum of these two outcomes is 1. In other words...
            $Prob(E) + Prob(\bar{E}) = 1$.
            
            For example, if $E$ is getting a 6 when rolling a die, then
            $Prob(E)$ is $\frac{1}{6}$. The likelihood of getting anything
            \textit{besides a 6} when rolling a die is $Prob(\bar{E}) = 1 - Prob(E)$,
            or $\frac{5}{6}$.

        \hrulefill

        \paragraph{Disjoint events:}
            Two events are disjoint if they cannot occur simultaneously.

            For example, you cannot roll one die and get \textit{both}
            a 1 and a 6.

        \paragraph{Independent events:}
            Two events are independent if the occurrence of one does not
            affect the probability of the other.

            For example, if we roll a die twice, the first roll doesn't
            affect the second roll.

            As another example, if we draw a card from a deck without replacement,
            this affects the second draw, because the size of the deck has
            been reduced by one.

        \paragraph{The General Sum Rule:}
            In an experiment, if we are trying to find the probability that \textit{either}
            $E_{1}$ OR $E_{2}$ occurs, we can find this by \textit{adding}
            together $Prob(E_{1})$ and $Prob(E_{2})$, as well as subtracting
            any overlap, $Prob(E_{1} \tab[0.1cm] AND \tab[0.1cm] E_{2})$.
            $$ Prob(E_{1} \tab[0.1cm] OR \tab[0.1cm] E_{2})
                = Prob(E_{1}) + Prob(E_{2}) - Prob(E_{1}\tab[0.1cm] AND \tab[0.1cm] E_{2})$$
            If $E_{1}$ and $E_{2}$ are \textbf{disjoint}, then this will be:
            $$ Prob(E_{1} \tab[0.1cm] OR \tab[0.1cm] E_{2})
                = Prob(E_{1}) + Prob(E_{2}) $$

        \paragraph{The General Product Rule:}
            In an experiment, if we are trying to find the probability
            that \textit{both} $E_{1}$ and $E_{2}$ occurs, we can solve this
            with
            $$ Prob(E_{1}\tab[0.1cm] AND \tab[0.1cm]E_{2})
                = Prob(E_{2}) \cdot Prob(E_{1} | E_{2}) $$
            If $E_{1}$ and $E_{2}$ are \textbf{independent}, then this will be:
            $$ Prob(E_{1}\tab[0.1cm] AND \tab[0.1cm]E_{2})
                = Prob(E_{1}) \cdot Prob(E_{2}) $$

        \paragraph{Bernoulli trial:}
            In a Bernoulli trial, we have some experiment where we repeat
            some experiment $n$ times. The success of the event we're
            checking in each run of this experiment has a probability $p$
            of success, and we are checking for exactly $k$ successes to
            occur.

            We can calculate the probability of $k$ successes occuring
            over $n$ runs of the experiment with:
            $$ C(n,k) \cdot p^{k} \cdot (1 - p)^{n-k} $$

        \paragraph{Expected (average) value:}
            Let's say for an experiment, a variable $X$ will receive
            some value randomly from the set $\{x_{1}, ..., x_{n}\}$.
            We can write the expected value (aka average value) as $E[X]$,
            and we can calculate this value with the sum...
            $$E[X] = (x_{1})Prob( X=x_{1} ) + ... + (x_{n})Prob(X = x_{n})$$

        \paragraph{Expected value for a Bernoulli trial:}
            If we're trying to find the expected value for a Bernoulli trial,
            we can compute it with the simpler formula:
            $$ E[X] = np $$
            Where $X$ is the amount of successful trials in the experiment.

        \paragraph{Amount of trials until first value:}
            If we want to run some trial continuously until the first time
            we receive some value (such as, flip a coin until we get the
            first HEADS value), we can calculate this.

            First, $X$ will be the amount of trials done (rolls, flips, etc.)
            until the first value is received, and $E[X]$ is the
            average amount of trials that get run.
            $p$ is the probability of success (getting the specified value).
            Then:
            $$ E[X] = p(1) + (1-p)(1 + E[X]) $$
            After plugging in values, you can solve for $E[X]$ algebraically.

        \paragraph{Matrix multiplication:}
            
        
    %------------------------------------------------------------------%
        \newpage
        \section*{Types of problems}

        \paragraph{Question 1 (6.1)} ~\\
            Experiment: Drawing a card from a deck of 52 cards.
            \begin{enumerate}
                \item[a.] What is the probability that the card is an Ace?
                \item[b.] What is the probability that the card is an Ace or a Queen?
                \item[c.] What is the probability that the card is Red or is a Jack?
            \end{enumerate}

        \paragraph{Question 2 (6.1)} ~\\
            List out the sample space set $S$ for each of the following,
            and give the size of the sample space $n(S)$.
            \begin{enumerate}
                \item[a.] Rolling a die.
                \item[b.] Flipping two coins.
            \end{enumerate}

        \paragraph{Question 3 (6.1)} ~\\
            We have a box that contains 12 kittens and 4 puppies.
            If we're selecting 5 pets, find the probability that...
            \begin{enumerate}
                \item[a.] All 5 pets are kittens.
                \item[b.] At most 1 pet is a puppy.
            \end{enumerate}

        \paragraph{Question 4 (6.1)} ~\\
            Experiment: Drawing two cards from a deck of 52 cards.
                What is the probability of getting two cards of the same value (and different suits)?
            \begin{enumerate}
                \item[a.] If we have 52 cards and we're selecting 2, what
                    structure type is this?
                \item[b.] For all items in the sample space $S$, what is the size $n(S)$ of the sample space?
                \item[c.] How many possibilities are there for selecting the first card?
                \item[d.] The second card is more restricted - it has to have the same
                    value as the first card. How many possibilities are there for a
                    second card that has the same value, but a different suit?
                \item[e.] What is the amount of events $n(E)$ in the experiment?
                \item[f.] What is the probability of a successful event $E$?
            \end{enumerate}

        \newpage

    %------------------------------------------------------------------%
        \newpage
        \section*{Answer key}
        
        \paragraph{Question 1 (6.1)}
            \begin{enumerate}
                \item[a.] 4 aces in a deck of 52, so $\frac{4}{52} = \frac{1}{13}$
                \item[b.] 4 Aces in the deck, and 4 Queens in the deck, so $\frac{4}{52} + \frac{4}{52} = \frac{8}{52} = \frac{2}{13}$
                \item[c.] Half the deck is red cards, and there are 4 jacks. There is an overlap
                    here of the 2 red jacks, so:
                    $\frac{26}{52} + \frac{4}{52} - \frac{2}{52}$
            \end{enumerate}
            
        \paragraph{Question 2 (6.1)}
            \begin{enumerate}
                \item[a.] $S = \{ 1, 2, 3, 4, 5, 6 \}$; \tab $n(S) = 6$
                \item[b.] $S = \{ HH, HT, TH, TT \}$; \tab $n(S) = 4$
            \end{enumerate}

        \paragraph{Question 3 (6.1)}
            \begin{enumerate}
                \item[a.] 12 kittens, select 5 / 16 pets, select 5 \tab
                    = $ \frac{C(12,5)}{C(16,5)} = \frac{33}{182} $
                \item[b.] Can have all kittens, or 4 kittens and 1 puppy. \\
                    5 kittens: $C(12,5)$ \tab{}
                    4 kittens, 1 puppy: $C(12,4) \cdot C(4,1)$ \\
                    = $ \frac{C(12,5) + C(12,4) \cdot C(4,1)}{C(16,5)} $
            \end{enumerate}

        \paragraph{Question 4 (6.1)}
            \begin{enumerate}
                \item[a.] Permutation, P(52,2)
                \item[b.] $n(S) = P(52,2) = 52 \cdot 51$
                \item[c.] 52
                \item[d.] 3
                \item[e.] $n(E) = 52 \cdot 3 = 156$
                \item[f.] $Prob(E) = \frac{n(E)}{n(S)} = \frac{52 \cdot 3}{52 \cdot 51} = \frac{1}{17}$
            \end{enumerate}

        \newpage


            
\end{document}
