\documentclass[a4paper,12pt]{book}
\usepackage[utf8]{inputenc}
\title{}
\author{Rachel Morris}
\date{\today}

\usepackage{rachwidgets}
\usepackage{fancyhdr}
\usepackage{lastpage}
\usepackage{dirtree}
\usepackage{boxedminipage}

\setcounter{chapter}{6}
\setcounter{section}{0}
\newcommand{\laChapter}{6.1 Introduction\ }

\newcommand{\laClass}{CS 211\ }
\newcommand{\laSemester}{Fall 2017\ }
\newcounter{question}

\pagestyle{fancy}
\fancyhf{}
\lhead{CS 211 Exercise}
\chead{Fall 2017}
\rhead{Ch \laChapter}
\rfoot{\thepage\ of \pageref{LastPage}}
\lfoot{\scriptsize Compiled by Rachel Morris, last updated \today}

\renewcommand{\headrulewidth}{2pt}
\renewcommand{\footrulewidth}{1pt}

\begin{document}

    %\toggletrue{answerkey}
    \togglefalse{answerkey}

    \notonkey{
    %- Team Info ------------------------------------------------------%

    \paragraph{Team name:}

    ~\\~\\
    Please write down all people in your team. ~\\

    % table %
    \begin{tabular}{ p{6cm} p{6cm} }
        1. & 2. \\ \\
        3. & 4.
    \end{tabular}
    % table %
    ~\\

    \hrulefill
    \subsection*{Grading}

    \begin{center}

        \begin{tabular}{ | l | l | l | }
            \hline
            \textbf{ Question } & \textbf{ Score } & \textbf{ Max } 
            \\ \hline
            1 &  & 1     \\ \hline
            2 &  & 1     \\ \hline
            3 &  & 1     \\ \hline
            4 &  & 1     \\ \hline
            5 &  & 3     \\ \hline
            6 &  & 3     \\ \hline
            7  &  & 3     \\ \hline
            8 &  & 3     \\ \hline
            & &  \\ \hline
            Total & & 16
            \\ \hline
        \end{tabular}
    \end{center}

\notonkey{ \newpage }{ \hrulefill }

    \section*{Review of structures}

    %------------------------------------------------------------------%

        \begin{introNOHEAD}{}
            \paragraph{Ordered lists of length $r$ with items from $\{1, ..., n\}$}
                \subparagraph{Repetitions allowed:}
                    An item from our input set $\{1, ..., n\}$
                    can be re-used multiple times for each selection.

                \subparagraph{Order matters:}
                    There is a difference between choosing item $A$ then $B$,
                    and choosing item $B$ then $A$. ~\\

                \begin{hint}{Example}
                    You are filling in your name in the high-score list,
                    and there are 3 slots to fill in your name. You can
                    repeat letters, and the order you enter them matters
                    (``RJM" is different from ``MRJ"). Assuming 26 letters,
                    the result is $26^{3}$.
                \end{hint}

            \hrulefill

            \paragraph{Unordered lists of length $r$ with items from $\{1, ..., n\}$}
                \subparagraph{Repetitions allowed:}
                    An item from our input set $\{1, ..., n\}$
                    can be re-used multiple times for each selection.

                \subparagraph{Order doesn't matter:}
                    Any grouping of selections from the set are considered
                    the same, such as \{a, b, c\} and \{b, c, a\}.

                \begin{hint}{Example}
                    The number of bags of $r$ pieces of fruit that can
                    be bought at a store with $n$ types of fruit available
                    is $C(r + n - 1, r)$.
                \end{hint}
        \end{introNOHEAD}
        }{}

        
        \notonkey{
        \newpage

        \begin{introNOHEAD}{}
            \paragraph{Permutations of length $r$ with items from $\{1, ..., n\}$}
                \subparagraph{No repetitions:}
                    Once one item is selected from the set $\{1, ..., n\}$,
                    it is no longer an option for subsequent items.

                    \begin{hint}{Example}
                        Pulling cards from a deck... First you have 52 options,
                        then 51 options, then 50 options...
                    \end{hint}

                \subparagraph{Order matters:}
                    The order that you select something matters, so
                    in a way, different ``slots" represent different things.

                    \begin{hint}{Example}
                        Electing President, VP, and Secretary
                    \end{hint}

            \hrulefill

            \paragraph{Combinations (Sets) of length $r$ with items from $\{1, ..., n\}$}
                \subparagraph{No repetitions:}
                    Once one item is selected from the set $\{1, ..., n\}$,
                    it is no longer an option for subsequent items.

                \subparagraph{Order doesn't matter:}
                    If any two items are selected, the order doesn't matter.
                    Combinations deal with sets, and with sets, \{a, b\} and
                    \{b, a\} are considered equivalent.

                \begin{hint}{Example}
                    If there are 5 different dinners, and we
                    need to feed 3 people, then there are $C(5,3)$ possible
                    dinner combinations.
                \end{hint}

        \end{introNOHEAD}
        }{}

        \newpage

    \notonkey{
        \begin{intro}{Types of structures}
            \begin{center}
                \begin{tabular}{l | c | c | c }
                    \textbf{}
                        & \textbf{Repeats}
                        & \textbf{Order}
                        & \textbf{}
                    \\
                    \textbf{Type}
                        & \textbf{allowed?}
                        & \textbf{matters?}
                        & \textbf{Formula}
                    \\ \hline
                    Ordered list of length $r$
                        & yes
                        & yes
                        & $n^{r}$

                    \\ \hline
                    Unordered list of length $r$
                        & yes
                        & no
                        & $C(r + n - 1, r)$
                    \\ \hline
                    Permutations of length $r$
                        & no
                        & yes
                        & $P(n,r) = \frac{n!}{(n-r)!}$
                    \\ \hline
                    Sets of length $r$
                        & no
                        & no
                        & $C(n,r) = \frac{n!}{r!(n-r)!}$
                \end{tabular}
            \end{center}
        \end{intro}
    }{}
    
        % -------------------------------------------------------------%
        % - QUESTION --------------------------------------------------%
        % -------------------------------------------------------------%
        \stepcounter{question}
        \begin{question}{\thequestion}{1}
            % 5.2 Exercise 12a
            How many arrangements are there of the letters in the word MATCH?

            \begin{itemize}
                \item[]     Are repetitions allowed?    \solution{ no }{ \fitb }
                \item[]     Does order matter?          \solution{ yes }{ \fitb }
                \item[]     What is $n$?                \solution{ 5 }{ \fitb }
                \item[]     What is $r$?                \solution{ 5 }{ \fitb }
                \item[]     Equation to use? $P(n,r)$ / $C(n,r)$ / $n^{r}$ / $C(r+n-1,r)$           \solution{ $P(n,r)$ }{ \fitb }
                \item[]     Solution: \solution{$P(5,5) = 120$}{\fitb[4cm]}
            \end{itemize}
        \end{question}

        \hrulefill

        % -------------------------------------------------------------%
        % - QUESTION --------------------------------------------------%
        % -------------------------------------------------------------%
        \stepcounter{question}
        \begin{question}{\thequestion}{1}
            % 5.3 Exercise 26 a
            There are five red, three green, and eight blue marbles in a box.
            In how many ways can a sample of four be selected?

            \begin{itemize}
                \item[]     Are repetitions allowed?    \solution{ yes }{ \fitb }
                \item[]     Does order matter?          \solution{ no }{ \fitb }
                \item[]     What is $n$?                \solution{ $5+3+8 = 16$ }{ \fitb }
                \item[]     What is $r$?                \solution{ 4 }{ \fitb }
                \item[]     Equation to use? $P(n,r)$ / $C(n,r)$ / $n^{r}$ / $C(r+n-1,r)$            \solution{ $C(n,r)$ }{ \fitb }
                \item[]     Solution: \solution{$C(16,4) = 1820$}{\fitb[4cm]}
            \end{itemize}
        \end{question}

        \hrulefill

        % -------------------------------------------------------------%
        % - QUESTION --------------------------------------------------%
        % -------------------------------------------------------------%
        \stepcounter{question}
        \begin{question}{\thequestion}{1}
            % POGIL 5.2 More Counting
            We can choose from four types of muffins:
            Blueberry, Orange, Chocolate Chip, or Cream Cheese.
            You're going to select muffins in this order:
            First for yourself, second for your sister,
            and third for your brother.
            It is OK if several people have the same muffin type.

            \begin{itemize}
                \item[]     Are repetitions allowed?    \solution{ yes }{ \fitb }
                \item[]     Does order matter?          \solution{ yes }{ \fitb }
                \item[]     What is $n$?                \solution{ 4 }{ \fitb }
                \item[]     What is $r$?                \solution{ 3 }{ \fitb }
                \item[]     Equation to use? $P(n,r)$ / $C(n,r)$ / $n^{r}$ / $C(r+n-1,r)$            \solution{ $n^{r}$ }{ \fitb }
                \item[]     Solution: \solution{$4^{3}$}{\fitb[4cm]}
            \end{itemize}
        \end{question}

        \hrulefill

        % -------------------------------------------------------------%
        % - QUESTION --------------------------------------------------%
        % -------------------------------------------------------------%
        \stepcounter{question}
        \begin{question}{\thequestion}{1}
            % Chapter 5.4 Exercise 2 b
            How many bags of 20 pieces of candy can one buy
            from a store that sells four types of candy?

            \begin{itemize}
                \item[]     Are repetitions allowed?    \solution{ yes }{ \fitb }
                \item[]     Does order matter?          \solution{ no }{ \fitb }
                \item[]     What is $n$?                \solution{ 20 }{ \fitb }
                \item[]     What is $r$?                \solution{ 4 }{ \fitb }
                \item[]     Equation to use? $P(n,r)$ / $C(n,r)$ / $n^{r}$ / $C(r+n-1,r)$            \solution{ $C(r+n-1,r)$ }{ \fitb }
                \item[]     Solution: \solution{$ C(20+4-1,20) = C(23,20) $}{\fitb[4cm]}
            \end{itemize}
        \end{question}


\notonkey{ \newpage }{ \hrulefill }

    \section{Introduction}
    %------------------------------------------------------------------%
    \subsection{Experiments, Outcomes, and Events}

    \notonkey{
        \begin{intro}{Vocabulary}

            For this chapter, we will be talking about experiments and their outcomes.
            For any given experiment, we will have a \textbf{sample space} of possible
            outcomes. This will be written as the set $S$.

            Within an experiment, we want to see if some \textbf{event} occurs,
            and how often it does. In cases where the event occurs, we call
            it a \textbf{success}.

            \paragraph{Definition}
            Given an experiment with a sample space $S$ of equally likely
            outcomes and an event $E$, the \textit{probability of the event}
            (denoted by $Prob(E)$) is the ratio of the number of successful
            outcomes to the total number of outcomes:
            \footnote{From Discrete Mathematics by Ensley and Crawley}

            $$Prob(E) = \frac{n(E)}{n(S)}$$

            (Recall that $n(S)$ is how we symbolically write, ``the amount
            of elements of the set $S$".)
        \end{intro}
    }{}
        % -------------------------------------------------------------%
        % - QUESTION --------------------------------------------------%
        % -------------------------------------------------------------%
        \stepcounter{question}
        \begin{question}{\thequestion}{3}
            Finish the following table to log all possible equally-likely
            outcomes for rolling a red four-sided die and a green four-sided die.

            \begin{center}
                \begin{tabular}{c | c c c c}
                    & \textbf{Green 1} & \textbf{Green 2} & \textbf{Green 3} & \textbf{Green 4}
                    \\ \hline
                    \textbf{Red 1}
                        & (1, 1)
                        & (1, 2)
                        & \solution{(1, 3)}{}
                        & \solution{(1, 4)}{}
                    \\
                    \textbf{Red 2}
                        & (2, 1)
                        & (2, 2)
                        & \solution{(2, 3)}{}
                        & \solution{(2, 4)}{}
                    \\
                    \textbf{Red 3}
                        & \solution{(3, 1)}{}
                        & \solution{(3, 2)}{}
                        & (3, 3)
                        & \solution{(3, 4)}{}
                    \\
                    \textbf{Red 4}
                        & \solution{(4, 1)}{}
                        & \solution{(4, 2)}{}
                        & \solution{(4, 1)}{}
                        & (4, 4)
                \end{tabular}
            \end{center}

            Using the definition above describe the following:

            \begin{enumerate}
                \item[a.] Both the red and green dice have the same values. ~\\
                    $n(E) = $ \solution{$4$}{\fitb}
                    \tab
                    $n(S) = $ \solution{$16$}{\fitb}
                    \tab
                    $Prob(E) = $ \solution{$\frac{4}{16} = \frac{1}{4}$}{\fitb}

                \item[b.] The sum of both dice values is $4$. ~\\
                    % (2,2), (1,3), (3,1)
                    $n(E) = $ \solution{$3$}{\fitb}
                    \tab
                    $n(S) = $ \solution{$16$}{\fitb}
                    \tab
                    $Prob(E) = $ \solution{$\frac{3}{16}$}{\fitb}
            \end{enumerate}

        \end{question}



        % -------------------------------------------------------------%
        % - QUESTION --------------------------------------------------%
        % -------------------------------------------------------------%
        \stepcounter{question} % Example 2
        \begin{question}{\thequestion}{3}
            Consider the experiment of drawing two cards from the top of
            a standard deck of 52 cards, and the event $E$ of the two cards
            having the same value.
            \footnote{From Discrete Mathematics by Ensley and Crawley}

            \begin{enumerate}
                \item[a.] Describe the set $S$ of all outcomes,
                    represented so that they are equally likely.

                \notonkey{
                    \begin{hint}{Hint}
                        This means what structure type is this? What
                        kind of formula are we using to choose 2 items
                        from a deck of 52?
                    \end{hint}
                }{}

                    $n(S) = $ \solution{ $P(52,2)$.}{ \fitb[3cm] }

                \item[b.] Describe the event $E$ in terms of your representation.

                \notonkey{
                    \begin{hint}{Hint}
                        We're interested in the event where both our selections
                        have the same value. This can be broken down as: \\
                        1. Choose any card (52 possible) \\
                        2. Choose a card with the same value (3 possible) \\
                        3. Combine with ``AND" (The Rule of Product)
                    \end{hint}
                }{}

                    $n(E) = $ \solution{ $(52)(3)$ }{}

                \item[c.] Compute $Prob(E) = \frac{n(E)}{n(S)}$.

                    $Prob(E) = \frac{n(E)}{n(S)} =$
                    \solution{ $ \frac{(52)(3)}{(52)(51)} $ }{}
            \end{enumerate}

        \end{question}

        \hrulefill

        % -------------------------------------------------------------%
        % - QUESTION --------------------------------------------------%
        % -------------------------------------------------------------%
        \stepcounter{question} % Practice Problem 2
        \begin{question}{\thequestion}{3}
            Consider the experiment of tossing a coin five successive
            times, and the event $E$ that the last two tosses have the same result. \\
            ( \fitb[0.5cm] \fitb[0.5cm] \fitb[0.5cm] Heads Heads ) OR
            ( \fitb[0.5cm] \fitb[0.5cm] \fitb[0.5cm] Tails Tails )

            \begin{enumerate}
                \item[a.] Describe the set $S$ of all outcomes, represented
                so they are equally likely
                \solution{ Ordered lists of length 5 with entries from $\{H, T\}$}{ ~\\ }

                \item[b.] Describe the event $E$ in terms of your representation. \\
                $n(S) = $ \solution{ $2^{5} = 32$ }{ \fitb } \tab
                $n(E) = $ \solution{ $2^{3} \cdot 1 \cdot 1$ + $2^{3} \cdot 1 \cdot 1$ = $16$ }{ \fitb }

                \item[c.] $Prob(E) = \frac{n(E)}{n(S)} = $
                    \solution{ $\frac{16}{32} = 0.5$ }{ }
            \end{enumerate}
        \end{question}

\notonkey{ \newpage }{ \hrulefill }

    \subsection{The complement of the Event}

\notonkey{
        \begin{intro}{Proposition 1}
            Given an event $E$,
            $$ Prob(E) + Prob(\bar{E}) = 1 $$
            Where $\bar{E}$ is the complement of the event $E$.
        \end{intro}
}{}

        % -------------------------------------------------------------%
        % - QUESTION --------------------------------------------------%
        % -------------------------------------------------------------%
        \stepcounter{question} % Example 3
        \begin{question}{\thequestion}{3}
            What is the probability that for a six-sided die rolled
            three times the same result comes up more than once?

            \begin{enumerate}
                \item[a.] What is the sample space $S$?
                    \solution{ \{1, 2, 3, 4, 5, 6\} }{ ~\\ }

                \item[b.] What is the event $E$ (in English)? ~\\
                    The set of outcomes that... \solution{ use the same \# more than once. }{ ~\\ }

                \item[c.] What is the complement of $\bar{E}$ (in English)? ~\\
                    The set of outcomes that... \solution{ are all different numbers. }{ ~\\ }

                \item[d.] What \textit{structure type} is $\bar{E}$? What is $n$ and $r$? ~\\
                    \solution{ Permutation, $n = 6$, $r = 3$ }{ ~\\ }

                \item[e.] Calculate $Prob(\bar{E})$ ~\\
                    $Prob(\bar{E}) = n(\bar{E}) / n(S) =$
                    \solution{ $\frac{P(6,3)}{6^{3}} = \frac{5}{9} = 0.\bar{5}$ }{}

                \item[f.] Calculate the probability for the Event $Prob(E)$ using the proposition. ~\\
                    \solution{ $1 - Prob(\bar{E}) = 1 - 0.55 \approx 0.44$ }{ ~\\ }
            \end{enumerate}
        \end{question}


\end{document}
