\documentclass[a4paper,12pt]{book}
\usepackage[utf8]{inputenc}

\usepackage{rachwidgets}


\newcommand{\laClass}       {CS 211}
\newcommand{\laSemester}    {Spring 2018}
\newcommand{\laChapter}     {5.3}
\newcommand{\laType}        {Exercise}
\newcommand{\laPoints}      {5}
\newcommand{\laTitle}       {Combinations}
\newcommand{\laDate}        {Jan 23, 2018}
\setcounter{chapter}{5}
\setcounter{section}{3}
\addtocounter{section}{-1}
\newcounter{question}

\toggletrue{answerkey}
\togglefalse{answerkey}


\title{}
\author{Rachel Singh}
\date{\today}

\pagestyle{fancy}
\fancyhf{}

\lhead{\laClass, \laSemester, \laDate}

\chead{}

\rhead{\laChapter\ \laType\ \iftoggle{answerkey}{ KEY }{}}

\rfoot{\thepage\ of \pageref{LastPage}}

\lfoot{\scriptsize By Rachel Singh, last updated \today}

\renewcommand{\headrulewidth}{2pt}
\renewcommand{\footrulewidth}{1pt}

\begin{document}




\footnotesize
~\\ 
\textbf{\laChapter\ \laType: } In-class exercises are meant to introduce you to a new topic
and provide some practice with the new topic. Work in a team of up to 4 people to complete this exercise.
You can work simultaneously on the problems, or work separate and then check your answers with each other.
Completion score is given for this assignment.

~\\
Team:\\
(1) \tab[6cm] (2) \\
(3) \tab[6cm] (4)

\hrulefill
\normalsize 


% ASSIGNMENT ------------------------------------ %

    \section{\laTitle}

    \subsection{Review}

    \begin{intro}{Permutation formula}
        A permutation is written as $P(n, r)$ where $n$ is the amount of
        items we have to choose from, and $r$ is the amount of items we
        are selecting. The formula for this is:

        $$ P(n, r) = \frac{n!}{(n-r)!} $$

        \paragraph{The Rule of Sums}
        
        \begin{quote}
            If we have A ways of doing something and B ways of doing
            another thing and we can not do both at the same time, then there are A + B ways to choose one of the actions.
            \footnote{From https://en.wikipedia.org/wiki/Rule\_of\_sum}
        \end{quote}

        \paragraph{The Rule of Complements}
        
        \begin{quote}
            If there are $x$ objects, and $y$ of those objects have a particular property,
            then the number of those objects that do \textbf{not} have that particular
            property is $x - y$.
            \footnote{From Discrete Mathematics, Ensley and Crawley, page 390}
        \end{quote}

        \paragraph{The Rule of Products}
        
        \begin{quote}
            If there are $a$ ways of doing something and $b$
            ways of doing another thing, then there are $a \cdot b$ ways of performing both actions.
            \footnote{From https://en.wikipedia.org/wiki/Rule\_of\_product}
        \end{quote}
        
    \end{intro}
    
    \subsection{Combinations and Permutations}

    \begin{intro}{Combinations}
        In mathematics, a combination is selection of items from a collection,
        such that (unlike permutations) the order of selection does not matter.
        (...) The number of $r$-combinations from a given set of $n$ elements
        is often denoted in elementary combinatorics texts by $C(n,r)$.
        \footnote{From https://en.wikipedia.org/wiki/Combinations}

        For a combination of length $r$ from a set of $n$ elements:
        $$C(n,r) = \frac{n!}{(n-r)! \cdot r!}$$

        (Note that the book uses $r$ and Wikipedia uses $k$.)
    \end{intro}
    

    \stepcounter{question}
    \begin{questionNOGRADE}{\thequestion}
        How many ways can you rearrange the letters in the word ``DOG''?
    \end{questionNOGRADE}
    
    \hrulefill
    
    \stepcounter{question}
    \begin{questionNOGRADE}{\thequestion}
        In how many ways can 10 children line up for lunch?
        \textit{Hint: Is this a combination problem or a permutation problem?
        Does order matter?}
    \end{questionNOGRADE}

    \hrulefill

    \stepcounter{question}
    \begin{questionNOGRADE}{\thequestion}
        In a class of 20 students, there are 7 IT majors and 13 CS majors.
        If 4 board positions had to be filled for the computer club,
        how many ways would there be to fill the positions with the given constraints?

        \begin{enumerate}
            \item[a.]   No constraints - any student can be on the board.
            \item[b.]   There must be exactly 2 IT students and 2 CS students on the board.
            \item[c.]   There must be \textit{at least} 2 IT students on the board.
        \end{enumerate}
    \end{questionNOGRADE}

    \newpage
    
    \stepcounter{question}
    \begin{questionNOGRADE}{\thequestion}
        Suppose we are going to receive a shipment of 50 games on floppy disk for our vintage game store.
        Each box of 50 generally has 3 defective floppies.
        For the shipment, we are going to select 5 games to feature in a display.

        \begin{enumerate}
            \item[a.]   How many total good floppies are there?
            \item[b.]   How many total bad floppies are there?
            \item[c.]   How many ways could we choose 5 games to feature?
            
            \item[d.]   How many ways contain \textit{no} defective floppies?
            
            \item[e.]   Using the Rule of Products, determine how many ways that contain all 3 defective floppies?
                    
            \item[f.]   Using the Rule of Sums and the Rule of Products, determine
                        how many ways contain \textit{at least one} defective floppy.

            \item[g.]   Using the Rule of Complements, determine
                        how many ways contain \textit{at least one} defective floppy.
                        The answer for (f) and (g) should match.
        \end{enumerate}
        
    \end{questionNOGRADE}

    \hrulefill

    \stepcounter{question}
    \begin{questionNOGRADE}{\thequestion}
        There's a bargain bin that has 5 PC games, 3 Playstation games, and 8 Xbox games.

        \begin{enumerate}
            \item[a.]   How many total games are there?
            
            \item[b.]   If you're grabbing 4 games to buy, you don't care about the order
                        that you pull the games out. How many ways can 4 games be selected?

            \item[c.]   How many ways can you select 4 games that are all for the same console?
                        \textit{(Hint: This means the Playstation games don't get counted.
                        Also, which Rule are you going to use to solve this?)}

            \item[d.]   How many selections of 4 games are there, where you have 2 for one platform,
                        and 2 for another platform?
                        \textit{(Hint: What Rules apply here?)}
        \end{enumerate}
    \end{questionNOGRADE}
    




\end{document}

