\input{BASE-HEAD}
\newcommand{\laClass}       {CS 211}
\newcommand{\laSemester}    {Spring 2018}
\newcommand{\laChapter}     {5.1}
\newcommand{\laType}        {Exercise}
\newcommand{\laPoints}      {5}
\newcommand{\laTitle}       {Intro to Combinatorics}
\newcommand{\laDate}        {Jan 16, 2018}
\setcounter{chapter}{5}
\setcounter{section}{1}
\addtocounter{section}{-1}
\newcounter{question}

\toggletrue{answerkey}


\title{}
\author{Rachel Singh}
\date{\today}

\pagestyle{fancy}
\fancyhf{}

\lhead{\laClass}

\chead{\laSemester}

\rhead{\laChapter\ \laType\ \iftoggle{answerkey}{ KEY }{}}

\rfoot{\thepage\ of \pageref{LastPage}}

\lfoot{\scriptsize By Rachel Morris, last updated \today}

\renewcommand{\headrulewidth}{2pt}
\renewcommand{\footrulewidth}{1pt}

\begin{document}




\notonkey{

\footnotesize
~\\ 
\textbf{\laChapter\ \laType: } In-class exercises are meant to introduce you to a new topic
and provide some practice with the new topic. Work in a team of up to 4 people to complete this exercise.
You can work simultaneously on the problems, or work separate and then check your answers with each other.
You can take the exercise home, score will be based on the in-class quiz the following class period.
\textbf{Work out problems on your own paper} - this document just has examples and questions.

\hrulefill
\normalsize 

}{
\begin{center}
    \Large
    \textbf{Answer Key}
\end{center}
}


\begin{enumerate}
    \item
        \begin{itemize}
            \item[a.]   Repetitions aren't allowed, order doesn't matter: Is a set (combination).
            \item[b.]   Repetitions aren't allowed, order does matter: Is a permutation.
            \item[c.]   Repetitions are allowed, order does matter: Is an ordered list.
            \item[d.]   Repetitions are allowed, order doesn't matter: Is an unordered list.
        \end{itemize}
        
    \item
        \begin{itemize}
            \item[a.]   
                \begin{tabular}{c | c c c c}
                    & A & B & C & D \\ \hline
                    A & (A,A) & (A,B) & (A,C) & (A,D)
                    \\
                    B & (B,A) & (B,B) & (B,C) & (B,D)
                    \\
                    C & (C,A) & (C,B) & (C,C) & (C,D)
                    \\
                    D & (D,A) & (D,B) & (D,C) & (D,D)
                \end{tabular}
                
            \item[b.]   
                \begin{tabular}{c | c c c c}
                    & A & B & C & D \\ \hline
                    A & (A,A) & (A,B) & (A,C) & (A,D)
                    \\
                    B &  & (B,B) & (B,C) & (B,D)
                    \\
                    C &  &  & (C,C) & (C,D)
                    \\
                    D &  &  &  & (D,D)
                \end{tabular}
        \end{itemize}
        
    \item
        \begin{itemize}
            \item[a.]   There are 36 total outomes, and 6 total doubles. 6/36 = 1/6
            \item[b.]   Double with a sum of less than 4 include just (1,1), so 1/36.
            \item[c.]   (1,5), (5,1), (5,2), (5,3), (5,4), (5,5), (5,6), (6,5).
                        So, 8
            \item[d.]   (1,5), (5,1), (5,2), (5,3), (5,4), (5,6), (6,5).
                        So, 7
        \end{itemize}
        
    \item
        \begin{itemize}
            \item[a.]   H, T
            \item[b.]   H-H, H-T, T-H, T-T
            \item[c.]   H-H-H, H-H-T, H-T-H, H-T-T, T-H-H, T-H-T, T-T-H, T-T-T
        \end{itemize}
        
    \item
        \begin{itemize}
            \item[a.]   4
            \item[b.]   They are equally likely
            \item[c.]   2
            \item[d.]   6
            \item[e.]   4
        \end{itemize}
        
        
\end{enumerate}
    

\input{BASE-FOOT}
