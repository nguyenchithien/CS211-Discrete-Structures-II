\documentclass[a4paper,12pt]{book}
\usepackage[utf8]{inputenc}

\usepackage{rachwidgets}


\newcommand{\laClass}       {CS 211}
\newcommand{\laSemester}    {Spring 2018}
\newcommand{\laChapter}     {6.4}
\newcommand{\laType}        {Exercise}
\newcommand{\laPoints}      {6}
\newcommand{\laTitle}       {Expected Value in Games of Chance}
\newcommand{\laDate}        {Mar 1, 2018}
\setcounter{chapter}{5}
\setcounter{section}{1}
\addtocounter{section}{-1}
\newcounter{question}

\toggletrue{answerkey}
\togglefalse{answerkey}


\title{}
\author{Rachel Singh}
\date{\today}

\pagestyle{fancy}
\fancyhf{}

\lhead{\laClass, \laSemester, \laDate}

\chead{}

\rhead{\laChapter\ \laType\ \iftoggle{answerkey}{ KEY }{}}

\rfoot{\thepage\ of \pageref{LastPage}}

\lfoot{\scriptsize By Rachel Singh, last updated \today}

\renewcommand{\headrulewidth}{2pt}
\renewcommand{\footrulewidth}{1pt}

\begin{document}




\footnotesize
~\\ 
\textbf{\laChapter\ \laType: } In-class exercises are meant to introduce you to a new topic
and provide some practice with the new topic. Work in a team of up to 4 people to complete this exercise.
You can work simultaneously on the problems, or work separate and then check your answers with each other.
Completion score is given for this assignment.

~\\
Team:\\
(1) \tab[6cm] (2) \\
(3) \tab[6cm] (4)

\hrulefill
\normalsize 


\notonkey{
% ASSIGNMENT ------------------------------------ %

    \section{\laTitle}

    \subsection{Expected Value}
    
        \begin{intro}{Definition}
            \footnotesize
            For a given probability experiment, let $X$ be a random
            variable whose possible values come from the set of numbers
            $ x_{1}, ..., x_{n} $. Then the \textbf{expected value of $X$},
            denoted by $E[X]$, is the sum

            $$ (x_{1}) \cdot Prob(X = x_{1}) + (x_{2}) \cdot Prob(X = x_{2}) + ... + (x_{n}) \cdot Prob(X = x_{n}) $$

            This is sometimes called the \textit{average value} of the random variable,
            thinking of the average of the values $X$ takes on over many repetitions of the experiment.
            \footnote{From Discrete Math by Ensley and Crawley, page 467}
            Suppose I have a ``loaded" die for which the probability of
            a 6 appearing is $\frac{1}{2}$, while the probability of each
            of the other faces appearing is $\frac{1}{10}$.
            What is the expected value on one roll? Compare to the expected
            value of a fair die.

            ~\\
            For the loaded die:

            $$E[X] = (1)(\frac{1}{10}) + (2)(\frac{1}{10}) + (3)(\frac{1}{10})
                    + (4)(\frac{1}{10}) + (5)(\frac{1}{10}) + (6)(\frac{1}{2})$$

            $$ = \frac{1}{10}(15) + \frac{1}{2}(6) = 4.5$$

            ~\\
            For the fair die:

            $$E[X] = (1)(\frac{1}{6}) + (2)(\frac{1}{6}) + (3)(\frac{1}{6})
                    + (4)(\frac{1}{6}) + (5)(\frac{1}{6}) + (6)(\frac{1}{6})$$

            $$ = \frac{1}{6}(21) = 3.5$$
        \end{intro}

    \newpage

    \stepcounter{question}
    \begin{questionNOGRADE}{\thequestion}
        Suppose you pay \$2 each time to play the following game: Two
        dice are rolled, and you win \$5 for each 6 that comes up.
        Do you expect to win more than you pay if you play many, many times?

        Let $X$ represent the amount of money you win in one play
        of the game. So, you can win either \$0, \$5, or \$10, so the
        values are $\{0, 5, 10\}$.

        ~\\ What is $Prob(X = 0)$? (The probability of getting no 6's?)
        ~\\ \vspace{2cm}

        ~\\ What is $Prob(X = 5)$? (The probability of getting one 6?)
        ~\\ \vspace{2cm}

        ~\\ What is $Prob(X = 10)$? (The probability of getting two 6's?)
        ~\\ \vspace{2cm}

        ~\\ Then, $E[X] = 0 \cdot Prob(X = 0) + 5 \cdot Prob(X = 5) + 10 \cdot (X = 10)$
        ~\\ \vspace{3cm}
    \end{questionNOGRADE}

    \newpage

    \stepcounter{question}
    \begin{questionNOGRADE}{\thequestion}
        Suppose that the payoff for the game (in question 1) is \$5
        for rolling one 6, and \$25 for rolling two 6's. Now is it worth \$2 to play?
        (Recalculate the expected value)
    \end{questionNOGRADE}

    \newpage
    
    \subsection{Expectation in Bernoulli trials}
    
        \begin{intro}{Theorem 1}
            Suppose an experiment consists of the independent repetition
            of a trial $n$ times, and the probability of that trial's
            individual success is $p$ each time it is performed.
            If $X$ denotes the number of successful trials in this experiment,
            then $E[X] = n \cdot p$.
            \footnote{From Discrete Math by Ensley and Crawley, page 469}

            \paragraph{Practice Problem 3}
            Use the definition of expected value to show that the average
            number of results of heads in an experiment
            consisting of tossing a coin three times is 1.5

            So, our $X$ will be the number of successful trials (heads tossed).
            The coin is tossed three times, so $X$ takes on the values from
            the set $\{0, 1, 2, 3\}$. By the definition of the expected value,

            ~\\
            $E[X] = \\ 0 \cdot Prob(X = 0) + 1 \cdot Prob(X = 1) + 2 \cdot Prob(X = 2)
            + 3 \cdot Prob(X = 3)$ ~\\~\\
            $= 0 \cdot C(3,0) \cdot (1/2)^{3} + 1 \cdot C(3,1) \cdot (1/2)^{3}
            + 2 \cdot C(3,2) \cdot (1/2)^{3} + 3 \cdot C(3,3) \cdot (1/2)^{3}$ ~\\~\\
            $ = (0 + 3 + 6 + 3) \cdot (1/8) = 1.5$

            ~\\
            And using the theorem, $n = 3$, $p = 1/2$...

            $E[X] = 3 \cdot (1/2) = (3/2) = 1.5$
        \end{intro}

    \newpage

    \stepcounter{question}
    \begin{questionNOGRADE}{\thequestion}
        If two teams, team Anteater and team Badger, play a
        best-of-three series, and if team Anteater has a (2/3)
        probability of winning any given game, then what is the
        average number of games in the series?

        ~\\
        Since it is a best-of-three match, there can be either
        two games (if one team wins the first and second match),
        or three games (if one team wins the first, the other the second, and either win the third).

        \begin{itemize}
            \item[a.]   What are the two possible values of $X$? (The amount of games played in the series?)
            \item[b.]   Using ``A" and ``B" to symbolize which team won, draw all the possible outcomes for only 2 games.
            \item[c.]   So what is the probability $Prob(X = 2)$? Use the sum rule to combine each outcome's probability.
            \item[d.]   For the other case with 3 games. If the series doesn't have 2 games, then the only other option is $Prob(X = 3) = 1 - Prob(X=2)$. What is $Prob(X=3)$?
            \item[e.]   Now calculate the value of $E[X]$ with the equation in Theorem 1.
        \end{itemize}
    \end{questionNOGRADE}


}{
% KEY ------------------------------------ %

    \begin{enumerate}
        \item
            \begin{itemize}
                \item   $Prob(X = 0) = \frac{5}{6} \cdot \frac{5}{6} = \frac{25}{36}$
                \item   $Prob(X = 5) = C(2,1)(\frac{1}{6}) \cdot \frac{5}{6} = \frac{10}{36}$
                \item   $Prob(X = 10) = \frac{1}{6} \cdot \frac{1}{6} = \frac{1}{36}$
                \item   $E[X] = 0 \cdot \frac{25}{36} + 5 \cdot \frac{10}{36} + 10 \cdot \frac{1}{36}$ \\
                        $ \tab = \frac{60}{36} $ \\
                        $ \tab \approx 1.67 $
            \end{itemize}

        \item   $E[X] = 0 \cdot Prob(X = 0) + 5 \cdot Prob(X = 5) + 25 \cdot Prob(X = 25)$
                ~\\
                $= 0 \cdot \frac{25}{36} + 5 \cdot \frac{10}{36} + 25 \cdot \frac{1}{36}$
                ~\\
                $= \frac{75}{36} \approx \$2.08${}
                ~\\
                You will win about 8 cents each time you play, over the long term.

        \item
            \begin{itemize}
                \item[a.]   2 or 3
                \item[b.]   AA or BB
                \item[c.]   $Prob(X = 2) = (2/3)^2 + (1/3)^2 = 5/9$
                \item[d.]   $Prob(X=3) = \frac{4}{9}$
                \item[e.]   $E[X] = 2 \cdot Prob(X=2) + 3 \cdot Prob(X=3) $ \\
                            $ \tab = 2 (5/9) + 3 (4/9) $ \tab
                            $ = (22/9) \approx 2.44$ games
            \end{itemize}
    \end{enumerate}

}



\end{document}

