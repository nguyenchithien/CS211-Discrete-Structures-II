\input{BASE-HEAD}
\newcommand{\laClass}       {CS 211}
\newcommand{\laSemester}    {Spring 2018}
\newcommand{\laChapter}     {5.}
\newcommand{\laType}        {Exercise}
\newcommand{\laPoints}      {5}
\newcommand{\laTitle}       {Solving Recurrence Relations}
\newcommand{\laDate}        {Jan 16, 2018}
\setcounter{chapter}{5}
\setcounter{section}{6}
\addtocounter{section}{-1}
\newcounter{question}

\toggletrue{answerkey}
\togglefalse{answerkey}


\title{}
\author{Rachel Singh}
\date{\today}

\pagestyle{fancy}
\fancyhf{}

\lhead{\laClass}

\chead{\laSemester}

\rhead{\laChapter\ \laType\ \iftoggle{answerkey}{ KEY }{}}

\rfoot{\thepage\ of \pageref{LastPage}}

\lfoot{\scriptsize By Rachel Morris, last updated \today}

\renewcommand{\headrulewidth}{2pt}
\renewcommand{\footrulewidth}{1pt}

\begin{document}




\notonkey{

\footnotesize
~\\ 
\textbf{\laChapter\ \laType: } In-class exercises are meant to introduce you to a new topic
and provide some practice with the new topic. Work in a team of up to 4 people to complete this exercise.
You can work simultaneously on the problems, or work separate and then check your answers with each other.
You can take the exercise home, score will be based on the in-class quiz the following class period.
\textbf{Work out problems on your own paper} - this document just has examples and questions.

\hrulefill
\normalsize 

}{
\begin{center}
    \Large
    \textbf{Answer Key}
\end{center}
}


% ASSIGNMENT ------------------------------------ %

    \section{\laTitle}

    \subsection{Review: Closed \& Recursive formulas}

    \begin{intro}{Remember chapter 1.2...?}
        Back in the first week of CS 210, you were given sequences of
        numbers... let's say

        $$ 3, 5, 7, 9, 11 $$

        ...and you tasked with coming up with \textbf{closed formulas}
        and \textbf{recursive formulas} for these.

        Closed formulas were based only on the value of $n$, such as...
        $$a_{n} = 2n + 1$$

        And recursive formulas were based on some starting value, $a_{1}$,
        with each subsequent element being based off a previous element., such as...
        $$ a_{1} = 3; \tab a_{n} = a_{n-1} + 2 $$
    \end{intro}

    A full class and 4.4 chapters later, we will actually show you \textit{how} to
    come up with a formula given a sequence of numbers...
    No guesswork required.

    \newpage

    \subsection{Difference tables}

    \begin{intro}{Finding formulas: Recursive}
        Find a recursive formula for the sequence $2, 5, 8, 11, 14, ...$

        ~\\     First, we will be assigning this sequence of numbers to a variable $s$,
                where $n$ is the index (or position) in the sequence, and $s_{n}$ is
                the element (or term) at that position.

        ~\\     Previously, we had our formulas begin at $a_{1}$, but now we will
                be starting our sequences at index $0$, like when programming.

        ~\\     First let's build a table with $n$, the index, $s_{n}$, the element
                at that index, and $\triangle_{n}$, the difference between two elements
                ($\triangle_{n} = s_{n+1} - s_{n}$)

        \begin{center}
            \begin{tabular}{| r | c | c | c | c | c | }
                \hline
                \footnotesize   index & & & & & \\ 
                $n$
                & 0 & 1 & 2 & 3 & 4
                \\ \hline
                
                \footnotesize element at $n$ & & & & & \\
                $s_{n}$
                & 2 & 5 & 8 & 11 & 14
                \\ \hline
                
                \footnotesize different between elements & & & & & \\
                $\triangle_{n}$
                & 3 & 3 & 3 & 3 & 3
                \\ \hline
            \end{tabular}
        \end{center}

        With the table, we can see the difference between each term (2 to 5, 5 to 8, etc...) is \textbf{3}.
        With this information, we can write any value $s_{1}$ in terms of the previous value:
    
        \begin{center}
            \begin{tabular}{c l | l}
                $s_{1}$ &   $= s_{0} + \triangle_{0}$                       & $(5 = 2 + 3)$
                \\ \\
                $s_{2}$ &   $= s_{1} + \triangle_{1}$                       & $(8 = 5 + 3)$
                \\
                &           $= s_{0} + (\triangle_{0} + \triangle_{1})$     & $(8 = 2 + 3 + 3)$
                \\ \\
                $s_{3}$ &   $= s_{2} + \triangle_{2}$                       & $(11 = 8 + 3)$
                \\
                &           $= s_{0} + (\triangle_{0} + \triangle_{1} + \triangle_{2})$ & $(11 = 2 + 3 + 3 + 3)$
                \\
                &           $= s_{0} + \sum_{k=0}^{2}{ \triangle_{k} }$     & $(11 = 2 + \sum_{k=0}^{2}{3})$
            \end{tabular}
        \end{center}

        \begin{center}
            So given the first term being \\
            $s_{0} = 2$,
            \\ we can say the recursive formula is \\
            $s_{n} = s_{n-1} + \triangle_{n-1}$ \tab or \tab
            $s_{n} = s_{n-1} + 3$
        \end{center}
        
    \end{intro}

    \newpage

    \begin{intro}{(Continued) Finding formulas: Closed}
        But what about the closed formula? Well with the differences
        we can also describe any term as the \textit{first term} plus
        the sum of the differences...
        
        \begin{center}
            \begin{tabular}{c l | l}
                $s_{1}$ &   $= s_{0} + \triangle_{0}$                       & $(5 = 2 + 3)$
                \\ \\
                $s_{2}$ &   $= s_{1} + \triangle_{1}$                       & $(8 = 5 + 3)$
                \\
                &           $= s_{0} + (\triangle_{0} + \triangle_{1})$     & $(8 = 2 + 3 + 3)$
                \\ \\
                $s_{3}$ &   $= s_{2} + \triangle_{2}$                       & $(11 = 8 + 3)$
                \\
                &           $= s_{0} + (\triangle_{0} + \triangle_{1} + \triangle_{2})$ & $(11 = 2 + 3 + 3 + 3)$
                \\
                &           $= s_{0} + \sum_{k=0}^{2}{ \triangle_{k} }$     & $(11 = 2 + \sum_{k=0}^{2}{3})$
            \end{tabular}
        \end{center}

        So $s_{1} = 2 + 3$, \tab $s_{2} = 2 + 3 + 3$, \tab $s_{3} = 2 + 3 + 3 + 3$...
        
        which we can write as

        $$ s_{n} = \sum_{k=0}^{n-1}{ (3) } + 2$$
        \begin{center}
            OR
        \end{center}
        $$ s_{n} = 3 \cdot n + 2 $$
    \end{intro}


    \begin{intro}{Theorem 1: Fundamental Theorem of Sums and Differences}
        For any sequence $\{s_{n}\}$ with first differences $\triangle_{k} = s_{k+1} - s_{k}$,
        and any $n \geq 1$,

        $$s_{n} - s_{0} = \sum_{k=0}^{n-1}{ \triangle_{k} } $$

            or
        
        $$s_{n} = \sum_{k=0}^{n-1}{ \triangle_{k} } + s_{0} $$
    \end{intro}

    \newpage
    
    \stepcounter{question}
    \begin{questionNOGRADE}{\thequestion}
        Build a difference table and find the formulas for the sequence:
        \footnote{From Jim Van Horn's POGIL exercises}

        $$ 7, 12, 17, 22, 27, ... $$
    \end{questionNOGRADE}

    \vspace{6cm}

    \stepcounter{question}
    \begin{questionNOGRADE}{\thequestion}
        Build a difference table and find the formulas for the sequence:
        \footnote{From Discrete Mathematics, 5.6 exercise 1a, Ensley and Crawley}

        $$ 3, 10, 17, 24, 31, ... $$
        
    \end{questionNOGRADE}

    \newpage

    \subsection{Complex sequences}

    \begin{intro}{Complex sequences}
        Sometimes, the difference between each term in a sequence is not the same;
        maybe the difference is 3, then 4, then 5, and so on. In this case,
        the difference itself also has a difference. In this case,
        that ``difference-of-the-differences" is known as the \textit{second difference},
        whereas the difference between the terms themselves is the \textit{first difference}.

        \paragraph{Example:} Find a closed formula for the sequence 6, 11, 19, 30, 44.

        As previously, we can start by writing out a table of the index $n$, the term $s_{n}$,
        and the difference $\triangle_{n}$...

        \begin{center}
            \begin{tabular}{l | l l l l l}
                $n$ & 0 & 1 & 2 & 3 & 4
                \\ \hline
                $s_{n}$ & 6 & 11 & 19 & 30 & 44
                \\ \hline
                $\triangle_{n}$ & 5 & 8 & 11 & 14 & ?
            \end{tabular}
        \end{center}

        Once we've figured out $\triangle_{n}$, we can see that it isn't
        constant each time, so we can't apply the same techniques as before.
        Instead, let's expand the table to have a fourth row: the difference
        of the differences. We will use a triangle again to symbolize ``difference",
        but we will add a number to it, so $\triangle_{n}^{1}$ is the difference
        between terms, and $\triangle_{n}^{2}$ is the difference of those
        differences.
        
        \begin{center}
            \begin{tabular}{l | l l l l l}
                $n$ & 0 & 1 & 2 & 3 & 4
                \\ \hline
                $s_{n}$ & 6 & 11 & 19 & 30 & 44
                \\ \hline
                $\triangle_{n}^{1}$ & 5 & 8 & 11 & 14 & ?
                \\ \hline
                $\triangle_{n}^{2}$ & 3 & 3 & 3 & ... & ...
            \end{tabular}
        \end{center}

        Ahh, can we maybe apply what we learned last time! Solving this will
        actually mean that we're working recursively.

        Let's look at this closer...

        \begin{center}
            \begin{tabular}{l | l l l l l}
                $n$ & 0 & 1 & 2 & 3 & 4
                \\ \hline
                $s_{n}$ & 6 & 11 & 19 & 30 & 44
                \\ \hline
                $\triangle_{n}^{1}$ & 5 & 8 & 11 & 14 & ?
                \\ \hline
                $\triangle_{n}^{2}$ & 3 & 3 & 3 & ... & ...
            \end{tabular}
        \end{center}

        $$\triangle_{n}^{k} = \triangle_{n+1}^{k-1} - \triangle_{n}^{k-1}$$
        
    \end{intro}

    \newpage


    \begin{intro}{(Continued) Complex sequences}
        
        \begin{center}
            \begin{tabular}{c l | l}
                $\triangle_{1}^{1}$
                &           $= \triangle_{0}^{1} + \triangle_{0}^{2}$
                & $(8 = 5 + 3)$
                \\
                $\triangle_{2}^{1}$
                &           $= \triangle_{1}^{1} + \triangle_{1}^{2}$
                & $(11 = 5 + 3 + 3)$
                \\
                &           $= \triangle_{0}^{1} + (\triangle_{0}^{2} + \triangle_{1}^{2})$
                & $(14 = 5 + 3 + 3)$
                \\
                $\triangle_{3}^{1}$
                &           $= \triangle_{2}^{1} + \triangle_{2}^{2}$
                & $(11 = 8 + 3)$
                \\
                &           $= \triangle_{0}^{1} + (\triangle_{0}^{2} + \triangle_{1}^{2} + \triangle_{2}^{2})$ & $(11 = 2 + 3 + 3 + 3)$
            \end{tabular}
    
        \end{center}

        So we can use the same strategy to find values of $\triangle_{n}^{1}$...

        $$ \triangle_{n}^{1} = \triangle_{0}^{1} + \sum_{k=0}^{n-1} ( \triangle_{k}^{2} ) $$

        $$ \triangle_{n}^{1} = 5 + \sum_{k=0}^{n-1} (3) $$

        $$ \triangle_{n}^{1} = 3n + 5 $$

        \paragraph{Finding $s_{n}$} ~\\

        After we have an equation for the first level difference, we can then repeat
        the Theorem to find $s_{n}$...

        $$ s_{n} = s_{0} + \sum_{k=0}^{n-1} ( \triangle_{k}^{1} ) $$

        $$ s_{n} = 6 + \sum_{k=0}^{n-1} (3k + 5) $$

        And simplifying the sum...
        
        $$ s_{n} = 6 + \sum_{k=0}^{n-1} (5) + 3 \sum_{k=0}^{n-1} (k) $$

        But how do we find the value of $\sum_{k=0}^{n-1} (k)$ ?
        
    \end{intro}

    \newpage

    \begin{intro}{Proposition 1 from Chapter 2.3}
        $$ \sum_{i=1}^{n} (i) = \frac{n(n+1)}{2} $$

        Or, rewritten for our use-case:

        $$ \sum_{k=0}^{n-1} (k) = \frac{n(n-1)}{2} $$        
    \end{intro}
    
    \begin{intro}{(Continued) Complex sequences}
        So, continuing to simplify, we have:
        $$ s_{n} = 6 + 5n + 3 \frac{n(n-1)}{2} $$
        $$ s_{n} = 6 + \frac{5n \cdot 2}{2} + \frac{3n(n-1)}{2} $$
        $$ s_{n} = \frac{10n + 3n^{2} - 3n}{2} + 6 $$
        $$ s_{n} = \frac{3}{2} n^{2} + \frac{7}{2} n + 6 $$

        And that's the final answer.
    \end{intro}
    
    \stepcounter{question}
    \begin{questionNOGRADE}{\thequestion}
        Build a difference table and find the \textbf{closed formula} for the sequence:
        \footnote{From Discrete Mathematics, 5.6 exercise 1c, Ensley and Crawley}

        $$ 1, 3, 8, 16, 27, 41, ... $$
        
    \end{questionNOGRADE}


    \newpage

    \subsection{Review Theorems}

    \begin{intro}{Definition of the $kth$ level difference at index $n$}
        $$\triangle_{n}^{k} = \triangle_{n+1}^{k-1} - \triangle_{n}^{k-1}$$

    \end{intro}
    
    \begin{intro}{Theorem 1 (Revisited)}
        $$s_{n} = \sum_{k=0}^{n-1}{ \triangle_{k}^{1} } + s_{0} $$
    \end{intro}

    \begin{intro}{Theorem 2:}
        $$s_{n} = s_{0} + \sum_{i=1}^{k}{ \triangle_{0}^{i} \cdot C(n,i) }$$
    \end{intro}

    
    \begin{intro}{Proposition 1 from Chapter 2.3}
        $$ \sum_{k=0}^{n-1} (k) = \frac{n(n-1)}{2} $$        
    \end{intro}

\input{BASE-FOOT}
