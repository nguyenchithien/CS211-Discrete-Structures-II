\documentclass[a4paper,12pt]{book}
\usepackage[utf8]{inputenc}

\usepackage{rachwidgets}


\newcommand{\laClass}       {CS 211}
\newcommand{\laSemester}    {Spring 2018}
\newcommand{\laChapter}     {5.5}
\newcommand{\laType}        {Exercise}
\newcommand{\laPoints}      {5}
\newcommand{\laTitle}       {Recursive Counting}
\newcommand{\laDate}        {Jan 16, 2018}
\setcounter{chapter}{5}
\setcounter{section}{5}
\addtocounter{section}{-1}
\newcounter{question}

\toggletrue{answerkey}
\togglefalse{answerkey}


\title{}
\author{Rachel Singh}
\date{\today}

\pagestyle{fancy}
\fancyhf{}

\lhead{\laClass, \laSemester, \laDate}

\chead{}

\rhead{\laChapter\ \laType\ \iftoggle{answerkey}{ KEY }{}}

\rfoot{\thepage\ of \pageref{LastPage}}

\lfoot{\scriptsize By Rachel Singh, last updated \today}

\renewcommand{\headrulewidth}{2pt}
\renewcommand{\footrulewidth}{1pt}

\begin{document}




\footnotesize
~\\ 
\textbf{\laChapter\ \laType: } In-class exercises are meant to introduce you to a new topic
and provide some practice with the new topic. Work in a team of up to 4 people to complete this exercise.
You can work simultaneously on the problems, or work separate and then check your answers with each other.
Completion score is given for this assignment.

~\\
Team:\\
(1) \tab[6cm] (2) \\
(3) \tab[6cm] (4)

\hrulefill
\normalsize 


    \section{\laTitle}

    \begin{intro}{Remember Chapter 2.3...?}
        \small 
        Back in CS 210, we did proofs by using induction. With induction,
        our goal was to show that some statement was true for the first
        value, $a_{1}$, and then for all values going up through $a_{m-1}$.
        Let's look at the steps for this sort of problem.
        
        \paragraph{Exercise 2.3, 3a from textbook}
        Show that the sequence defined by $a_{k} = a_{k-1} + 4; a_{1} = 1$ for
        $k \geq 2$ is equivalently described by the closed formula $a_{n} = 4n-3$.

        \subparagraph{Step 1:} Check values for both formulas, for $a_{1}$:

            \tab Recursive: $a_{1} = 1$ (provided)
            \tab Closed: $a_{1} = 4(1) - 3 = 1$

            They match, so we can continue.

        \subparagraph{Step 2:} Rewrite the recursive formula in terms of $m$:
            $a_{m} = a_{m-1} + 4$

        \subparagraph{Step 3:} Find the equation for $a_{m-1}$ through the recursive formula:

            \tab $a_{n} = 4n - 3$ \\
            \tab $=>$ $a_{m-1} = 4(m-1) - 3$ \\
            \tab $=>$ $a_{m-1} = 4m - 7$

        \subparagraph{Step 4:} Plug $a_{m-1}$ into the recursive formula from step 2, and simplify.

            \tab $a_{m} = a_{m-1} + 4$ \\
            \tab $=>$   $a_{m} = (4m-7) + 4$ \\
            \tab $=>$   $a_{m} = 4m - 3$

        \subparagraph{PROOF:} $a_{m} = 4m-3$ and the closed formula $a_{n} = 4n-3$ match,
            so the closed formula and recursive formula are equivalent.

    \end{intro}

    This section is quite small and just highlighting how we can use
    our previous proofs with Permutations.

    \newpage
    
    \subsection{Recursive counting}

    \begin{intro}{Proving for recurrence relations}
        Prove by induction and the recurrence relation
        \footnote{From 5.5 Exercise 4, Discrete Mathematics by Ensley and Crawley}
        
        $P(n,r) = n \cdot P((n-1), (r-1))$ with $P(n,0)=1$
            
        \begin{center}
            that $P(n,n) = n!$ for all $n \geq 0$.
        \end{center}
        
        
        \subparagraph{Step 1:} Check that it works for $n=1$: \\
            $$P(1,1) = 1 \cdot P(0, 0)$$
            $$ = \frac{0!}{0!} = \frac{1}{1} = 1;$$
            $$ 1! = 1 $$
            This step is OK! Moving on...

        \subparagraph{Step 2:}  Plug in $n$ for $r$ in the original statement. 
            $$P(n,n) = n \cdot P( (n-1), (n-1) )$$
            $$= n \cdot \frac{(n-1)!}{( (n-1) - (n-1) )!}$$
            $$= n \cdot \frac{(n-1)!}{( n - 1 - n + 1 )!}$$
            $$= n \cdot \frac{(n-1)!}{( n - n )!}$$
            $$= n \cdot \frac{(n-1)!}{( 0 )!}$$
            $$= n \cdot \frac{(n-1)!}{1}$$
            $$= n \cdot (n-1)!$$
            $$= n!$$
            ($n$ times $(n-1)!$ is equivalent to just having $n!$)
            
    \end{intro}

    \begin{intro}{Review: Permutation}
        $$P(n,r) = \frac{n!}{(n-r)!}$$
    \end{intro}
    
    \stepcounter{question}
    \begin{questionNOGRADE}{\thequestion}
        Use the recurrence relation

        $$P(n,r) = n \cdot P((n-1), (r-1))$$
        (with $P(n,0) = 1$ for all $n \geq 0$)
        to prove each of the following:
        \footnote{From 5.5 Exercise 5, Discrete Mathematics by Ensley and Cralwey}

        \begin{itemize}
            \item[a.]   For all $n \geq 1$, $P(n,1) = n$

            \item[b.]   For all $n \geq 2$, $P(n,2) = n \cdot (n-1)$

            \item[c.]   For all $n \geq 3$, $P(n,3) = n \cdot (n-1) \cdot (n-2)$
        \end{itemize}
    \end{questionNOGRADE}





\end{document}

