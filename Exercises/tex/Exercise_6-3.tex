\input{BASE-HEAD}
\newcommand{\laClass}       {CS 211}
\newcommand{\laSemester}    {Spring 2018}
\newcommand{\laChapter}     {6.3}
\newcommand{\laType}        {Exercise}
\newcommand{\laPoints}      {6}
\newcommand{\laTitle}       {Probability in Games of Chance}
\newcommand{\laDate}        {Feb 27, 2018}
\setcounter{chapter}{6}
\setcounter{section}{1}
\addtocounter{section}{-1}
\newcounter{question}

\toggletrue{answerkey}
\togglefalse{answerkey}


\title{}
\author{Rachel Singh}
\date{\today}

\pagestyle{fancy}
\fancyhf{}

\lhead{\laClass}

\chead{\laSemester}

\rhead{\laChapter\ \laType\ \iftoggle{answerkey}{ KEY }{}}

\rfoot{\thepage\ of \pageref{LastPage}}

\lfoot{\scriptsize By Rachel Morris, last updated \today}

\renewcommand{\headrulewidth}{2pt}
\renewcommand{\footrulewidth}{1pt}

\begin{document}




\notonkey{

\footnotesize
~\\ 
\textbf{\laChapter\ \laType: } In-class exercises are meant to introduce you to a new topic
and provide some practice with the new topic. Work in a team of up to 4 people to complete this exercise.
You can work simultaneously on the problems, or work separate and then check your answers with each other.
You can take the exercise home, score will be based on the in-class quiz the following class period.
\textbf{Work out problems on your own paper} - this document just has examples and questions.

\hrulefill
\normalsize 

}{
\begin{center}
    \Large
    \textbf{Answer Key}
\end{center}
}


\notonkey{
% ASSIGNMENT ------------------------------------ %

    \section{\laTitle}

    \subsection{Bernoulli Trials}
    
        \begin{intro}{Theorem 1}
            Given a simple experiment, called a \textbf{Bernoulli trial},
            and an event that occurs with a probability $p$, if the trial is repeated
            independently $n$ times, then the probability of having exactly
            $k$ successes is
            \footnote{From Discrete Math by Ensley and Crawley, page 460}
            $$ C(n, k) \cdot p^{k} \cdot (1 - p)^{n-k} $$

            \paragraph{Example 1}
            What is the probability that in 10 successive rolls of a fair, six-sided
            die, we get exactly five results of 6?

            ~\\
            Here, we have $n = 10$, $k = 5$, and $p = \frac{1}{6}$, so:

            $$C(10, 5) \cdot (\frac{1}{6})^{5} \cdot (1 - \frac{1}{6})^{10-5}$$

            $$\frac{10!}{5!(10-5)!} \cdot (\frac{1}{6})^{5} \cdot (\frac{5}{6})^{5} $$

            $$\frac{3628800}{14400} \cdot \frac{1}{7776} \cdot \frac{3125}{7776} $$

            \begin{center}
                $ \approx 0.013 $
            \end{center}
        \end{intro}

    \newpage

    \stepcounter{question}
    \begin{questionNOGRADE}{\thequestion}
        % Exercise 1

        What is the probability of getting exactly 3 heads
        on 10 tosses of a fair coin?
        
        \begin{center}
            \includegraphics[width=10cm]{images/6-3-coins.png}
        \end{center}

        \vspace{1cm}
        
        $n$, the amount of trial repeats:
        
        \vspace{1cm}
        
        $k$, the amount of successes (heads):
        
        \vspace{1cm}
        
        $p$, the probability of success:

        ~\\~\\
        Use the formula of $ C(n, k) \cdot p^{k} \cdot (1 - p)^{n-k} $
        to find the probability.
    \end{questionNOGRADE}

    \newpage

        
        % -------------------------------------------------------------%
        % - QUESTION --------------------------------------------------%
    \stepcounter{question}
    \begin{questionNOGRADE}{\thequestion}
        % Exercise 2
        What is the probability that in seven rolls of a six-sided die,
        the result of 1 appears \textit{at least} five times?

        \begin{center}
            \includegraphics[width=12cm]{images/6-3-dice.png}
        \end{center}

        \begin{hint}{Hint}
            For this one, we will need to use the \textbf{rule of sums}
            to combine several outcomes: Getting 5 1's, 6 1's, OR 7 1's.
        \end{hint}

        \large
        \begin{center}
            \begin{tabular}{ | c | p{4cm} | c | c | c | }
                \hline
                & & repeats $n$ & successes $k$ & probability $p$
                \\ \hline
                A &
                Getting five 1's
                    & \solution{ $ 7 $ }{ 7 }
                    & \solution{ $ 5 $ }{ 5 }
                    & \solution{ 1/6 }{ 1/6 }
                \\ \hline
                B &
                Getting six 1's
                    & \solution{ $ 7 $ }{ 7 }
                    & \solution{ $ 6 $ }{}
                    & \solution{ 1/6 }{}
                \\ \hline
                C &
                Getting seven 1's
                    & \solution{ $ 7 $ }{ 7 }
                    & \solution{ $ 7 $ }{}
                    & \solution{ 1/6 }{}
                \\ \hline
            \end{tabular}
        \end{center}
        \normalsize

        Now, using the formula $ C(n, k) \cdot p^{k} \cdot (1 - p)^{n-k} $
        three different times for case (A), (B), and (C).

        ~\\ (A) \tab $ C(n, k) \cdot p^{k} \cdot (1 - p)^{n-k} $ =
        ~\\ \vspace{1cm}
        
        ~\\ (B) \tab $ C(n, k) \cdot p^{k} \cdot (1 - p)^{n-k} $ =
        ~\\ \vspace{1cm}
        
        ~\\ (C) \tab $ C(n, k) \cdot p^{k} \cdot (1 - p)^{n-k} $ =
        ~\\ \vspace{1cm}

        To find the probability of getting at least five 1's in seven rolls,
        add (A), (B), and (C) together. (Just write out the formula; don't solve.)

        ~\\
        $Prob($ at least five 1's $) = $
    \end{questionNOGRADE}

        \notonkey{ \newpage }{ \hrulefill }
        
        % -------------------------------------------------------------%
        % - QUESTION --------------------------------------------------%
        % -------------------------------------------------------------%
        \stepcounter{question}
        \begin{question}{\thequestion}{3}
            % Exercise 5
            What is the probability of getting exactly one 6 on 10 tosses
            of a fair six-sided die?            
        \end{question}


}{
% KEY ------------------------------------ %

    \begin{enumerate}
        \item
            \begin{itemize}
                \item   $n = 10$
                \item   $k = 3$
                \item   $p = 1/2$
                \item   $ C(10, 3) \cdot (1/2)^{3} \cdot (1/2)^{7} = \frac{15}{128}$
            \end{itemize}

        \item
            \begin{itemize}
                \item   
                    \begin{tabular}{ | c | p{4cm} | c | c | c | }
                        \hline
                        & & repeats $n$ & successes $k$ & probability $p$
                        \\ \hline
                        A & Getting five 1's & 7 & 5 & 1/6
                        \\ \hline
                        B & Getting six 1's & 7 & 6 & 1/6
                        \\ \hline
                        C & Getting seven 1's & 7 & 7 & 1/6
                        \\ \hline
                    \end{tabular}
                
                \item[A.]   $ C(n, k) \cdot p^{k} \cdot (1 - p)^{n-k} $ = $C(7,5) \cdot (1/6)^{5} \cdot (5/6)^{2}$
                \item[B.]   $ C(n, k) \cdot p^{k} \cdot (1 - p)^{n-k} $ = $C(7,6) \cdot (1/6)^{6} \cdot (5/6)^{1}$
                \item[C.]   $ C(n, k) \cdot p^{k} \cdot (1 - p)^{n-k} $ = $C(7,7) \cdot (1/6)^{7} \cdot (5/6)^{0}$
                \item   $Prob($ at least five 1's $) = $
                        $C(7,5) \cdot (1/6)^{5} \cdot (5/6)^{2} +
                        C(7,6) \cdot (1/6)^{6} \cdot (5/6)^{1} +
                        C(7,7) \cdot (1/6)^{7} \cdot (5/6)^{0}$
            \end{itemize}
        \item   $n = 10$, $k = 1$, $p = (1/6)$ \\
                $C(10,1) \cdot (1/6)^{1} \cdot (5/6)^{9} \approx 0.323 $
    \end{enumerate}

}

\input{BASE-FOOT}
