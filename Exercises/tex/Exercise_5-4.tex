\documentclass[a4paper,12pt]{book}
\usepackage[utf8]{inputenc}

\usepackage{rachwidgets}


\newcommand{\laClass}       {CS 211}
\newcommand{\laSemester}    {Spring 2018}
\newcommand{\laChapter}     {5.4}
\newcommand{\laType}        {Exercise}
\newcommand{\laPoints}      {5}
\newcommand{\laTitle}       {Binary Sequences}
\newcommand{\laDate}        {Jan 25, 2018}
\setcounter{chapter}{5}
\setcounter{section}{4}
\addtocounter{section}{-1}
\newcounter{question}

\toggletrue{answerkey}
\togglefalse{answerkey}


\title{}
\author{Rachel Singh}
\date{\today}

\pagestyle{fancy}
\fancyhf{}

\lhead{\laClass, \laSemester, \laDate}

\chead{}

\rhead{\laChapter\ \laType\ \iftoggle{answerkey}{ KEY }{}}

\rfoot{\thepage\ of \pageref{LastPage}}

\lfoot{\scriptsize By Rachel Singh, last updated \today}

\renewcommand{\headrulewidth}{2pt}
\renewcommand{\footrulewidth}{1pt}

\begin{document}




\footnotesize
~\\ 
\textbf{\laChapter\ \laType: } In-class exercises are meant to introduce you to a new topic
and provide some practice with the new topic. Work in a team of up to 4 people to complete this exercise.
You can work simultaneously on the problems, or work separate and then check your answers with each other.
Completion score is given for this assignment.

~\\
Team:\\
(1) \tab[6cm] (2) \\
(3) \tab[6cm] (4)

\hrulefill
\normalsize 


\notonkey{
% ASSIGNMENT ------------------------------------ %

    \section{\laTitle}
    
    \subsection{Binary sequences}

    \begin{intro}{Binary sequence theorem}
        \begin{quote}
			The number of binary sequences with $r$ 1's and $n-r$ 0's is $C(n,r)$ or $C(n, n-r)$.
			\footnote{From Discrete Mathematics, Ensley and Crawley, page 409}
        \end{quote}
        
        This can be used for straightforward problems, like ``how many binary sequences are there
        with $x$ 1's and $y$ 0's?'', but we can also use it for more practical applications...!
    \end{intro}

    \stepcounter{question}
    \begin{questionNOGRADE}{\thequestion}
        How many binary sequences are there with three 1's and two 0's?
        
        \begin{enumerate}
	        \item[a.]	How many 1's? $r = $
	        \item[b.]	How many 0's? $n-r = $
	        \item[c.]	What is the value of $r$?
	        \item[d.]	How many binary sequences are there? $C(n,r) = $
		\end{enumerate}
    \end{questionNOGRADE}
    
    \hrulefill
       
    \stepcounter{question}
    \begin{questionNOGRADE}{\thequestion}
	    Using the Binary Sequence theorem and the Sum Rule, find the number
	    of binary sequences that have an odd number of 1's.
	\end{questionNOGRADE}
	
	\newpage
	
	\begin{intro}{Other uses of Binary Sequences}
		We can also use this theorem for strings that have more than just 0's and 1's...
		
		\paragraph{Example 2 from the book:} How many ordered lists of 10 letters, chosen from $\{m, a, t\}$, 
		have exactly three $m$'s?
		
		If we think in terms of spots to fill for each letter, we can diagram it like this:
		
		\begin{center}
			\begin{tabular}{c c c c c c c c c c}
				\fitb[0.5cm] & \fitb[0.5cm] & \fitb[0.5cm] & \fitb[0.5cm] & \fitb[0.5cm] & \fitb[0.5cm] & \fitb[0.5cm] & \fitb[0.5cm] & \fitb[0.5cm] & \fitb[0.5cm]
				\\
				1 & 2 & 3 & 4 & 5 & 6 & 7 & 8 & 9 & 10
			\end{tabular}
		\end{center}
		
		~\\ Since we have a restriction on the $m$'s, these should be filled in first. Selecting three $m$'s gives us
		$C(10,3) = 120$ combinations.
		
		~\\ Then, since there are no restrictions on the rest of the letters, we can select the final 7.
		However, we don't have 3 to choose from anymore; we wanted \textbf{exactly} 3 $m$'s and we have
		already filled those. So, instead of selecting from 3 options, we only have two: $\{a, t\}$.
		
		~\\ Since for each of the remaining, we have two options each, it will be calculated as $2 \cdot 2 \cdot 2 \cdot ...$
		seven times. These remaining 2's are an \textbf{ordered list}: Order matters, and we can have duplicates ($a$ or $t$).
		
		~\\ Our final result will be $C(10,3) \cdot 2^{7}$.
	\end{intro}
	
    \begin{intro}{Formulas for each structure type}
	
		\begin{center}
			\begin{tabular}{l | c | c | c }
				\textbf{}
					& \textbf{Repeats}
					& \textbf{Order}
					& \textbf{}
				\\
				\textbf{Type}
					& \textbf{allowed?}
					& \textbf{matters?}
					& \textbf{Formula}
				\\ \hline
				Ordered list of length $r$
					& yes
					& yes
					& $n^{r}$

				\\ \hline
				Unordered list of length $r$
					& yes
					& no
					& $C(r + n - 1, r)$
				\\ \hline
				Permutations of length $r$
					& no
					& yes
					& $P(n,r) = \frac{n!}{(n-r)!}$
				\\ \hline
				Sets of length $r$
					& no
					& no
					& $C(n,r) = \frac{n!}{r!(n-r)!}$
			\end{tabular}
		\end{center}
        
    \end{intro}
	
	\newpage
	
    \stepcounter{question}
    \begin{questionNOGRADE}{\thequestion}
	    How many distinguishable arrangements are there of the letters in the word ``BANANA''?
	    
	    \begin{enumerate}
		    \item[a.]	How many B's?		\tab How many A's?		\tab How many N's?
		    \item[b.]	How many ways are there to select a position for B?
						\textit{(Specify the formula and the final answer)}
			\item[c.]	How many ways are there to place the A's?
						\textit{(Hint: B has been placed, so there are 5 available spaces and 3 A's to place.)}
			\item[d.]	How many ways are there to place the N's?
						\textit{(Hint: How many spaces are left? How many N's to place?}
			\item[e.]	Which Rule do you use to find the final answer? What is the final answer?
		\end{enumerate}
	\end{questionNOGRADE}
	
	\hrulefill
	
    \stepcounter{question}
    \begin{questionNOGRADE}{\thequestion}
	    How many distinguishable arrangements are there of the word \\ ``PENNSYLVANIA''?
	    \textit{(Hint: You need the total amount of letters, and a count for each letter in the word!)}
	\end{questionNOGRADE}
	
}{
% KEY ------------------------------------ %
    \begin{itemize}
        \item[1a.]	$r = 3$
        \item[1b.]	$n-r = 2$
        \item[1c.]	$n-3 = 2$ \tab $n = 2 + 3$ \tab $n = 5$
        \item[1d.]	$C(n,r) = C(5,3) = 10$
        
        \item[2.]	Length 5 means 1x, 3x, or 5x ones.
					\\ one 1, four 0:	$r = 1$, $n-r = 4$, $n-1=4, n=5$, $C(5,1) = 5$
					\\ three 1, two 0:	$r = 3$, $n-r = 2$, $n-3=2, n=5$, $C(5,3) = 10$
					\\ five 1, zero 0:	$r = 5$, $n-r = 0$, $n-5=0, n=5$, $C(5,5) = 1$
					\\ $5 + 10 + 1 = 16$
		
		\item[3a.]	1x B, 3x A, 2x N.
		\item[3b.]	There are $C(6,1) = 6$ ways to place the B.
		\item[3c.]	There are $C(5,3) = 10$ ways to place the A's.
		\item[3d.]	There are $C(2,2) = 1$ ways to place the N's.
		\item[3e.]	Use the Rule of Products: $6 \cdot 10 \cdot 1 = 60$.
		
		\item[4.]	P: $C(12,1)$, E: $C(11,1)$, N: $C(10,3)$, S: $C(7,1)$, Y: $C(6,1)$, L: $C(5,1)$, V: $C(4,1)$, I: $C(3,1)$, A: $C(2,2)$ \\
					= $12 \cdot 11 \cdot 120 \cdot 7 \cdot 6 \cdot 5 \cdot 4 \cdot 3 \cdot 1$ \\
					= 39,916,800
    \end{itemize}
}



\end{document}

