\documentclass[a4paper,12pt]{book}
\usepackage[utf8]{inputenc}

\usepackage{rachwidgets}


\newcommand{\laClass}       {CS 211}
\newcommand{\laSemester}    {Spring 2018}
\newcommand{\laChapter}     {6.1}
\newcommand{\laType}        {Exercise}
\newcommand{\laPoints}      {6}
\newcommand{\laTitle}       {Intro to Probability}
\newcommand{\laDate}        {Feb 20, 2018}
\setcounter{chapter}{6}
\setcounter{section}{1}
\addtocounter{section}{-1}
\newcounter{question}

\toggletrue{answerkey}
\togglefalse{answerkey}


\title{}
\author{Rachel Singh}
\date{\today}

\pagestyle{fancy}
\fancyhf{}

\lhead{\laClass, \laSemester, \laDate}

\chead{}

\rhead{\laChapter\ \laType\ \iftoggle{answerkey}{ KEY }{}}

\rfoot{\thepage\ of \pageref{LastPage}}

\lfoot{\scriptsize By Rachel Singh, last updated \today}

\renewcommand{\headrulewidth}{2pt}
\renewcommand{\footrulewidth}{1pt}

\begin{document}




\footnotesize
~\\ 
\textbf{\laChapter\ \laType: } In-class exercises are meant to introduce you to a new topic
and provide some practice with the new topic. Work in a team of up to 4 people to complete this exercise.
You can work simultaneously on the problems, or work separate and then check your answers with each other.
Completion score is given for this assignment.

~\\
Team:\\
(1) \tab[6cm] (2) \\
(3) \tab[6cm] (4)

\hrulefill
\normalsize 


\notonkey{
% ASSIGNMENT ------------------------------------ %

    \section{\laTitle}

    \subsection{Experiments, Outcomes, and Events}

        \begin{intro}{Vocabulary}

            For this chapter, we will be talking about experiments and their outcomes.
            For any given experiment, we will have a \textbf{sample space} of possible
            outcomes. This will be written as the set $S$.

            Within an experiment, we want to see if some \textbf{event} occurs,
            and how often it does. In cases where the event occurs, we call
            it a \textbf{success}.

            \paragraph{Definition}
            Given an experiment with a sample space $S$ of equally likely
            outcomes and an event $E$, the \textit{probability of the event}
            (denoted by $Prob(E)$) is the ratio of the number of successful
            outcomes to the total number of outcomes:
            \footnote{From Discrete Mathematics by Ensley and Crawley}

            $$Prob(E) = \frac{n(E)}{n(S)}$$

            (Recall that $n(S)$ is how we symbolically write, ``the amount
            of elements of the set $S$".)
        \end{intro}

    \newpage

    \stepcounter{question}
    \begin{questionNOGRADE}{\thequestion}
        Finish the following table to log all possible equally-likely
        outcomes for rolling a red six-sided die and a green six-sided die.

        \large 
        \begin{center}
            \begin{tabular}{c | c c c c c c}
                & \textbf{G 1} & \textbf{G 2} & \textbf{G 3} & \textbf{G 4} & \textbf{G 5} & \textbf{G 6}
                \\ \hline
                \textbf{R 1}
                    & (1, 1)
                    & (1, 2)
                \\
                \textbf{R 2}
                    & (2, 1)
                    & (2, 2)
                \\
                \textbf{R 3}
                    & & & (3, 3)
                \\
                \textbf{R 4}
                    & & & &(4, 4)
                \\
                \textbf{R 5}
                    & & & & & (5, 5)
                \\
                \textbf{R 6}
                    & & & & & & (6, 6)
            \end{tabular}
        \end{center}
        \normalsize

        Using the definition above describe the following:

        \begin{enumerate}
            \item[a.] Both the red and green dice have the same values.
            
                \vspace{1cm}
                
                $n(E) =$ {\fitb}
                \tab
                $n(S) =$ {\fitb}
                \tab
                $Prob(E) = $ {\fitb}

            ~\\
            \item[b.] The sum of both dice values is $4$.

                \vspace{1cm}
                
                % (2,2), (1,3), (3,1)
                $n(E) = $ {\fitb}
                \tab
                $n(S) = $ {\fitb}
                \tab
                $Prob(E) = $ {\fitb}

            ~\\
            \item[c.]   \textbf{Exactly} one die has the value ``4''.

                \vspace{1cm}
                
                % (2,2), (1,3), (3,1)
                $n(E) = $ {\fitb}
                \tab
                $n(S) = $ {\fitb}
                \tab
                $Prob(E) = $ {\fitb}

            ~\\
            \item[d.]   \textbf{At least} one die has the value ``4''.

                \vspace{1cm}
                
                % (2,2), (1,3), (3,1)
                $n(E) = $ {\fitb}
                \tab
                $n(S) = $ {\fitb}
                \tab
                $Prob(E) = $ {\fitb}
        \end{enumerate}
    \end{questionNOGRADE}

    \newpage

    \stepcounter{question}
    \begin{questionNOGRADE}{\thequestion}
        For a scenario where we roll two dice, fill out the table with
        all possible roll outcomes that give a sum from the minimum sum, 2,
        to the maximum sum, 12.
    \end{questionNOGRADE}

    \Large
    \begin{center}
        \begin{tabular}{c | p{8cm} | l }
            \textbf{Dice sum} & \textbf{Outcomes} & \textbf{$n(E)$} \\ \hline
            2 & \textit{(1, 1)}
            \\ \hline
            3 & \textit{(2, 1), (1, 2)}
            \\ \hline
            4 &
            \\ \hline
            5 &
            \\ \hline
            6 &
            \\ \hline
            7 &
            \\ \hline
            8 &
            \\ \hline
            9 &
            \\ \hline
            10 &
            \\ \hline
            11 &
            \\ \hline
            12 & \textit{(6, 6)}
            \\ \hline
            $n(S)$ &
            \\ \hline
        \end{tabular}
    \end{center}
    \normalsize

    \begin{itemize}
        \item[a.]   How many total outcomes are there for rolling two dice? \vspace{1cm}   
        \item[b.]   What is the probability that rolling two dice will result in a sum of 12? \vspace{1cm}
        \item[c.]   What is the probability that rolling two dice will result in a sum of 8? \vspace{1cm}
        \item[d.]   What is the probability that rolling two dice will result in a sum greater than 7? \vspace{1cm}
    \end{itemize}
    
    \newpage

    \stepcounter{question} % Example 2
        \begin{questionNOGRADE}{\thequestion}
            Consider the experiment of drawing two cards from the top of
            a standard deck of 52 cards, and the event $E$ of the two cards
            having the same value.
            \footnote{From Discrete Mathematics by Ensley and Crawley}

            \begin{enumerate}
                \item[a.]   If we're drawing two cards from a deck of 52, what is the \textbf{structure}?
                            What is $n$ and $r$?
                            This will be the size of the sample space, $n(S)$.
                            \vspace{1cm}

                \item[b.]   Use combinatorics to figure out the amount of outcomes, $n(E)$, that result in $E$ being successful.
                            (i.e., how do calculate the amount of outcomes where two cards have the same values?)

                            \begin{center}
                                \begin{tabular}{ c c c }
                                    \fitb & & \fitb
                                    \\
                                    \footnotesize Any card & & \footnotesize Any card with same value
                                    \\
                                    52 & $\times$ & 3
                                \end{tabular}
                            \end{center}

                            $n(E) = $
                            \vspace{1cm}

                \item[c.] Compute $Prob(E) = \frac{n(E)}{n(S)}$. ~\\

                    $Prob(E) = \frac{n(E)}{n(S)} =$
            \end{enumerate}

        \end{questionNOGRADE}

    \newpage


        \stepcounter{question} % Practice Problem 2
        \begin{questionNOGRADE}{\thequestion}
            Consider the experiment of tossing a coin five successive
            times, and the event $E$ that the last two tosses have the same result. \\
            \begin{center}
            ( \fitb[0.5cm] \fitb[0.5cm] \fitb[0.5cm] Heads Heads ) OR
            ( \fitb[0.5cm] \fitb[0.5cm] \fitb[0.5cm] Tails Tails )
            \end{center}

            \begin{enumerate}
                \item[a.]   $n$ = amount of outcomes =  ~\\~\\
                            $r$ = amount of trials = ~\\~\\
                            Structure is = \\
                            \footnotesize
                            (combination / permutation / ordered list / unordered list)
                            \normalsize ~\\~\\
                            $n(S) = $
                            

                \item[b.]   How many outcomes are there where the last two coins result in heads? \vspace{2cm}

                \item[c.]   How many outcomes are there where the last two coints result in tails? \vspace{2cm}

                \item[d.]   What rule do we use to get the total amount of outcomes that match our event $E$? 
                            \footnotesize
                            (rule of sums / rule of products)
                            \normalsize ~\\~\\

                \item[e.]   $n(E) = $ \vspace{1cm}

                \item[f.]   $Prob(E) = \frac{n(E)}{n(S)} = $
            \end{enumerate}
        \end{questionNOGRADE}

    \subsection{The Complement of the Event}
        \begin{intro}{Proposition 1}
            Given an event $E$,
            $$ Prob(E) + Prob(\bar{E}) = 1 $$
            Where $\bar{E}$ is the complement of the event $E$.
        \end{intro}
        
        \stepcounter{question} % Example 3
        \begin{questionNOGRADE}{\thequestion}
            What is the probability that for a six-sided die rolled
            three times the same result comes up more than once?

            \begin{enumerate}
                \item[a.]   What is the sample space $S$? ~\\~\\
                            What is the size of the sample space, $n(S)$? ~\\~\\

                \item[b.]   If $E$ is the event where we have a number show up more than once,
                            what is the complement $\bar{E}$ of this event? ~\\~\\


                \item[c.]   What is the value of $n$, the amount of options?   ~\\ ~\\ ~\\ 
                            What is the value of $r$, the amount of trials?    ~\\ ~\\ ~\\ 
                            What is the structure type for rolling a die?      ~\\ ~\\ ~\\ 
                            What is the size of the set of outcomes for $\bar{E}$ (i.e., $n(\bar{E})$)? ~\\ ~\\ ~\\ 

                \item[d.] Calculate $Prob(\bar{E})$,  $Prob(\bar{E}) = n(\bar{E}) / n(S) =$ ~\\

                \item[e.] Calculate the probability for the Event $Prob(E)$ using the proposition.
            \end{enumerate}
        \end{questionNOGRADE}


}{
% KEY ------------------------------------ %

    \begin{enumerate}
        \item   
            \begin{tabular}{c | c c c c c c}
                & \textbf{G 1} & \textbf{G 2} & \textbf{G 3} & \textbf{G 4} & \textbf{G 5} & \textbf{G 6}
                \\ \hline
                \textbf{R 1}
                    & (1, 1) & (1, 2) & (1, 3) & (1, 4) & (1, 5) & (1, 6)
                \\
                \textbf{R 2}
                    & (2, 1) & (2, 2) & (2, 3) & (2, 4) & (2, 5) & (2, 6)
                \\
                \textbf{R 3}
                    & (3, 1) & (3, 2) & (3, 3) & (3, 4) & (3, 5) & (3, 6)
                \\
                \textbf{R 4}
                    & (4, 1) & (4, 2) & (4, 3) & (4, 4) & (4, 5) & (4, 6)
                \\
                \textbf{R 5}
                    & (5, 1) & (5, 2) & (5, 3) & (5, 4) & (5, 5) & (5, 6)
                \\
                \textbf{R 6}
                    & (6, 1) & (6, 2) & (6, 3) & (6, 4) & (6, 5) & (6, 6)
            \end{tabular}
            ~\\
            \begin{itemize}
                \item[a.]   $n(E) = 6$, $n(S) = 6 \cdot 6 = 36$, $Prob(E) = \frac{6}{36} = \frac{1}{6}$
                \item[b.]   Outcomes are (2, 2), (3, 1), and (1, 3). \\
                            $n(E) = 3$, $n(S) = 36$, $Prob(E) = \frac{3}{36} = \frac{1}{12}$
                \item[c.]   Outcomes are (1, 4), (2, 4), (3, 4), (5, 4), (6, 4), (4, 1), (4, 2), (4, 3), (4, 5), (4, 6) \\
                            $n(E) = 10$, $n(S) = 36$, $Prob(E) = \frac{10}{36} = \frac{5}{18}$
                \item[d.]   Outcomes are (1, 4), (2, 4), (3, 4), (4, 4) (5, 4), (6, 4), (4, 1), (4, 2), (4, 3), (4, 5), (4, 6) \\
                            $n(E) = 11$, $n(S) = 36$, $Prob(E) = \frac{11}{36}$
            \end{itemize}

        \item   
            \begin{tabular}{c | p{8cm} | l }
                \textbf{Dice sum} & \textbf{Outcomes} & \textbf{$n(E)$} \\ \hline
                2 & \textit{(1, 1)}                                     & 1
                \\ \hline
                3 & \textit{(2, 1), (1, 2)}                             & 2
                \\ \hline
                4 & (1, 3), (2, 2), (3, 1)                              & 3
                \\ \hline
                5 & (1, 4), (2, 3), (3, 2), (4, 1)                      & 4
                \\ \hline
                6 & (1, 5), (2, 4), (3, 3), (4, 2), (5, 1)              & 5
                \\ \hline
                7 & (1, 6), (2, 5), (3, 4), (4, 3), (5, 2), (6, 1)      & 6
                \\ \hline
                8 & (2, 6), (3, 5), (4, 4), (5, 3), (6, 2)              & 5
                \\ \hline
                9 & (3, 6), (4, 5), (5, 4), (6, 4)                      & 4
                \\ \hline
                10 & (4, 6), (5, 5), (6, 4)                             & 3
                \\ \hline
                11 & (5, 6), (6, 5)                                     & 2
                \\ \hline
                12 & \textit{(6, 6)}                                    & 1
                \\ \hline
            \end{tabular} ~\\
            \begin{itemize}
                \item[a.]   $n(S) = 36$
                \item[b.]   $n(E) = 1$, $Prob(E) = \frac{1}{36}$
                \item[c.]   $n(E) = 5$, $Prob(E) = \frac{5}{36}$
                \item[d.]   $n(E) = 5 + 4 + 3 + 2 + 1 = 15$, $Prob(E) = \frac{15}{36}$
            \end{itemize}

        \item
            \begin{itemize}
                \item[a.]   Permutation, $P(52, 2)$ is the sample space ($n(S)$).
                \item[b.]   $n(E) = 52(3) = 156$
                \item[c.]   $\frac{(52)(3)}{(52)(51)}$
            \end{itemize}

        \item
            \begin{itemize}
                \item[a.]   This is an ordered list problem. $n = 2$, $r = 5$, so $n(S) = 2^{5} = 32$.
                \item[b.]   End in heads: $2 \cdot 2 \cdot 2 \cdot 1 \cdot 1 = 8$
                \item[c.]   End in tails: $2 \cdot 2 \cdot 2 \cdot 1 \cdot 1 = 8$
                \item[d.]   Use the rule of sums.
                \item[e.]   $n(E) = 8 + 8 = 16$
                \item[f.]   $Prob(E) = \frac{n(E)}{n(S)} = \frac{16}{32} = \frac{1}{2}$
            \end{itemize}

        \item
            \begin{itemize}
                \item[a.]   $S = \{1, 2, 3, 4, 5, 6\}$, $n(S) = 6$.
                \item[b.]   $\bar{E} = $ All results are different numbers.
                \item[c.]   Permutation, $n = 6$, $r = 3$
                \item[d.]   $Prob(\bar{E}) = n(\bar{E}) / n(S) = \frac{P(6,3)}{6^{3}} = \frac{5}{9} = 0.\bar{5}$
                \item[e.]   $Prob(E) = 1 - Prob(\bar{E}) = 1 - 0.55 \approx 0.44$
            \end{itemize}
    \end{enumerate}

}



\end{document}

