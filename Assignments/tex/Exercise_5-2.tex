\documentclass[a4paper,12pt]{book}
\usepackage[utf8]{inputenc}

\usepackage{rachwidgets}


\newcommand{\laClass}       {CS 211}
\newcommand{\laSemester}    {Spring 2017}
\newcommand{\laChapter}     {5.2}
\newcommand{\laType}        {Exercise}
\newcommand{\laPoints}      {5}
\newcommand{\laTitle}       {Basic Rules for Counting}
\newcommand{\laDate}        {Jan 18, 2018}
\setcounter{chapter}{5}
\setcounter{section}{1}
\addtocounter{section}{-1}
\newcounter{question}

\toggletrue{answerkey}
\togglefalse{answerkey}


\title{}
\author{Rachel Singh}
\date{\today}

\pagestyle{fancy}
\fancyhf{}

\lhead{\laClass, \laSemester, \laDate}

\chead{}

\rhead{\laChapter\ \laType\ \iftoggle{answerkey}{ KEY }{}}

\rfoot{\thepage\ of \pageref{LastPage}}

\lfoot{\scriptsize By Rachel Singh, last updated \today}

\renewcommand{\headrulewidth}{2pt}
\renewcommand{\footrulewidth}{1pt}

\begin{document}




\footnotesize
~\\ 
\textbf{\laChapter\ \laType: } In-class exercises are meant to introduce you to a new topic
and provide some practice with the new topic. Work in a team of up to 4 people to complete this exercise.
You can work simultaneously on the problems, or work separate and then check your answers with each other.
Completion score is given for this assignment.

~\\
Team:\\
(1) \tab[6cm] (2) \\
(3) \tab[6cm] (4)

\hrulefill
\normalsize 


\notonkey{
% ASSIGNMENT ------------------------------------ %

    \section{\laTitle}


    \subsection{Permutations}

    \begin{intro}{Permutation formula}
        A permutation is a type of structure that describes counting when
        \textbf{order matters} and \textbf{repetitions are not allowed}.

        ~\\  A permutation is written as $P(n, r)$ where $n$ is the amount of
        items we have to choose from, and $r$ is the amount of items we
        are selecting. The formula for this is:

        $$ P(n, r) = \frac{n!}{(n-r)!} $$

        ~\\ Where $n!$ is $n$-factorial. Also note that $0! = 1$.
    \end{intro}
    

    \stepcounter{question}
    \begin{questionNOGRADE}{\thequestion}
        Calculate the following by hand using the formula:

        \begin{enumerate}
            \item[a.]   $P(5, 2)$
            \item[b.]   $P(5, 5)$
            \item[c.]   $P(5, 1)$
        \end{enumerate}
    \end{questionNOGRADE}

    \stepcounter{question}
    \begin{questionNOGRADE}{\thequestion}
        Use \texttt{www.wolframalpha.com} or a graphing calculator to compute the following:

        \begin{enumerate}
            \item[a.]   $P(26, 3)$
            \item[b.]   $P(52, 6)$
            \item[c.]   $P(26, 2) + P(10, 2)$
        \end{enumerate}
    \end{questionNOGRADE}

    \newpage
    
    \subsection{The Rule of Sums}

    \begin{intro}{The Rule of Sums}
        \begin{quote}
            In combinatorics, the rule of sum or addition principle is a basic counting principle.
            Stated simply,
            \textbf{it is the idea that if we have A ways of doing something and B ways of doing
            another thing and we can not do both at the same time, then there are A + B ways to choose one of the actions.}
            \footnote{From https://en.wikipedia.org/wiki/Rule\_of\_sum}
        \end{quote}
    \end{intro}

    \stepcounter{question}
    \begin{questionNOGRADE}{\thequestion}
        Uttam wants to read a new book series. He can either pick up
        the whole series of ``Native Tongue", which is \textit{3 books},
        or the whole series
        of ``Seed to Harvest", which is \textit{4 books}.
        How many ways can Uttam select books, if he is choosing 2 Native Tongue books, OR
        2 Seed to Harvest books, but not both?
    \end{questionNOGRADE}

    \hrulefill

    \stepcounter{question}
    \begin{questionNOGRADE}{\thequestion}
        Jennifer is trying to set up their class schedule so it won't
        interfere with work. They can fit in the following:
        \begin{itemize}
            \item Morning classes: 3 per week
            \item Night classes: 3 per week
            \item Weekend classes: 4 per week
        \end{itemize}
        Jennifer can only take all morning, all night, or all weekend classes -
        they cannot mix them.
        If there are 12 classes to choose from, how many combinations of classes can Jennifer have?       
    \end{questionNOGRADE}
        
        \begin{hint}{\ }
            When questions are phrased as ``can choose \textbf{either ... OR ...}",
            this usually points to \textbf{adding} the size of both sets.
        \end{hint}

    \newpage
    
    \begin{intro}{The rule of sums with overlap}
        If the list to count can be split into two pieces of size $z$
        and $y$, and the pieces have $z$ objects in common, then the original
        list has $x + y - z$ entries. In terms of sets, we can write this as
        $n(A \cup B) = n(A) + n(B) - n(A \cap B)$ for all sets $A$ and $B$.
    \footnote{From Discrete Math Mathematical Reasoning and Proofs with Puzzles, Patterns and Games, by Ensley and Crawley}
    \end{intro}

    \subsection{The Rule of Complements}
    
    \begin{intro}{The rule of complements}
        \begin{quote}
        If there are $x$ objects, and $y$ of those objects have a particular property,
        then the number of those objects that do \textbf{not} have that particular
        property is $x - y$.
        \footnote{From Discrete Mathematics, Ensley and Crawley, page 390}
        \end{quote}
    \end{intro}
    
    \stepcounter{question}
    \begin{questionNOGRADE}{\thequestion}
        Assume you are rolling two dice.

        \begin{enumerate}
            \item[a.]   List out all possible results, assuming order does matter.
            \begin{center}
                \Large 
                \begin{tabular}{ c | p{1cm} p{1cm} p{1cm} p{1cm} p{1cm} p{1cm} }
                    & 1 & 2 & 3 & 4 & 5 & 6
                    \\ \hline
                    1 \\ 2 \\ 3 \\ 4 \\ 5 \\ 6 
                \end{tabular}
            \end{center}

            \item[b.]   How many total outcomes are there?

            \item[c.]   How many results have 6 show up at least once?

            \item[d.]   Use the rule of complements to solve how many
                        outcomes there are where 6 does not show up.
        \end{enumerate}
    \end{questionNOGRADE}

    \stepcounter{question}
    \begin{questionNOGRADE}{\thequestion}
            Ryan has 107 games in his Steam library.
            Of those, 32 have the category ``action",
            14 have the category ``RPG",
            and 8 are categorized as both ``action" AND ``RPG", so they end up getting double-counted.

        \begin{enumerate}
            \item[a.] How many games are action or RPG? \\(Use the rule of sums with overlap)
            \item[b.] How many games are NOT action or RPG? \\(Use the rule of complements)
        \end{enumerate}        
    \end{questionNOGRADE}

    \hrulefill

    \subsection{The Rule of Products}

    \begin{intro}{The rule of products}
        \begin{quote}
        In combinatorics, the rule of product or multiplication principle is a
        basic counting principle (a.k.a. the fundamental principle of counting).
        Stated simply,
        \textbf{it is the idea that if there are $a$ ways of doing something and $b$
        ways of doing another thing, then there are $a \cdot b$ ways of performing both actions.}
        \footnote{From https://en.wikipedia.org/wiki/Rule\_of\_product}
        \end{quote}
    \end{intro}

    \stepcounter{question}
    \begin{questionNOGRADE}{\thequestion}
        On an old arcade machine, the highscore entries allow for
        3 letters for a player to sign their name. Many only offer capital letters,
        so only 26 options, and each letter can be used more than once.
        
        \begin{enumerate}
            \item[a.]   How many options are there for the first letter?
            \item[b.]   How many options are there for the second letter?
            \item[c.]   How many options are there for the third letter?
            \item[d.]   If you're choosing a first letter \textbf{and} a second letter \textbf{and} a third letter,
                        how many possible combinations are there?
        \end{enumerate}
    \end{questionNOGRADE}
    
}{
% KEY ------------------------------------ %
    \begin{itemize}
        \item[1a.]  20
        \item[1b.]  120
        \item[1c.]  5
        \item[2a.]  15,600
        \item[2b.]  14,658,134,400
        \item[2c.]  740
        \item[3.]   $P(3,2) + P(4,2) = 6 + 12 = 18$
        \item[4.]   $P(12,3) + P(12,3) + P(6,4) = 1,320 + 1,320 + 360 = 3,000$
        
        \item[5a.]  
                    \begin{tabular}{ c | c c c c c c}
                        & 1 & 2 & 3 & 4 & 5 & 6
                        \\ \hline
                        1 & 1-1 & 1-2 & 1-3 & 1-4 & 1-5 & 1-6
                        \\
                        2 & 2-1 & 2-2 & 2-3 & 2-4 & 2-5 & 2-6
                        \\
                        3 & 3-1 & 3-2 & 3-3 & 3-4 & 3-5 & 3-6
                        \\
                        4 & 4-1 & 4-2 & 4-3 & 4-4 & 4-5 & 4-6
                        \\
                        5 & 5-1 & 5-2 & 5-3 & 5-4 & 5-5 & 5-6
                        \\
                        6 & 6-1 & 6-2 & 6-3 & 6-4 & 6-5 & 6-6
                    \end{tabular}
        \item[5b.]  36
        \item[5c.]  11: (1,6), (2,6), (3,6), (4,6), (5,6), (6,6), (6,5), (6,4), (6,3), (6,2), (6,1)
        \item[5d.]  36 - 11 = 25
        
        \item[6a.]  32 + 14 - 8 = 38
        \item[6b.]  107 - 38 = 69
        
        \item[7a.]  26
        \item[7b.]  26
        \item[7c.]  26
        \item[7b.]  $26 \cdot 26 \cdot 26 = 17,576$
    \end{itemize}
}



\end{document}

