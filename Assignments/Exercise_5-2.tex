\documentclass[a4paper,12pt]{book}
\usepackage[utf8]{inputenc}
\title{}
\author{Rachel Morris}
\date{\today}

\usepackage{rachwidgets}
\usepackage{fancyhdr}
\usepackage{lastpage}
\usepackage{dirtree}
\usepackage{boxedminipage}

\newcommand{\laChapter}{5.2\ }

\pagestyle{fancy}
\fancyhf{}
\lhead{CS 211}
\chead{Fall 2017}
\rhead{Ch \laChapter Exercise}
\rfoot{\thepage\ of \pageref{LastPage}}
\lfoot{\scriptsize By Rachel Morris, last updated \today}

\renewcommand{\headrulewidth}{2pt}
\renewcommand{\footrulewidth}{1pt}

\begin{document}

    %\toggletrue{answerkey}
    \togglefalse{answerkey}

    %------------------------------------------------------------------%
    %- INSTRUCTIONS ---------------------------------------------------%
    %------------------------------------------------------------------%
    
    \chapter*{Chapter \laChapter In-class Exercise} \stepcounter{chapter}
    
    %------------------------------------------------------------------%
    %- Exercise Begin -------------------------------------------------%
    %------------------------------------------------------------------%

    \begin{center}
        \textbf{This is the instruction page, 
        make sure to fill out your answers in the \textbf{worksheet} later in the document}
    \end{center}

    %------------------------------------------------------------------%
    \section*{1. The rule of sums}

        \begin{intro}{The rule of sums}
        In combinatorics, the rule of sum or addition principle is a basic counting principle. Stated simply, it is the idea that if we have A ways of doing something and B ways of doing another thing and we can not do both at the same time, then there are A + B ways to choose one of the actions.

        More formally, the rule of sum is a fact about set theory. It states that sum of the sizes of a finite collection of pairwise disjoint sets is the size of the union of these sets. That is, if
        $S_{1}, S_{2}, ..., S_{n}$ are pairwise disjoint sets, then we have:

        $|S_{1}| + |S_{2}| + ... + |S_{n}| = |S_{1} \cup S_{2} \cup ... \cup S_{n}|$
        \footnote{From https://en.wikipedia.org/wiki/Rule\_of\_sum}
        \end{intro}

        % - QUESTION --------------------------------------------------%
        \begin{question}{1}{10\%}
            Uttam wants to read a new book series. He can either pick up
            the whole series of ``Native Tongue", which is \textit{3 books},
            or the whole series
            of ``Seed to Harvest", which is \textit{4 books}. How many books are there total
            that Uttam could read?
        \end{question}

        \hrulefill

        \newpage
        % - QUESTION --------------------------------------------------%
        \begin{question}{2}{10\%}
            Jennifer is trying to set up their class schedule so it won't
            interfere with work. They can fit in the following:
            \begin{itemize}
                \item Morning classes: 3 per week
                \item Night classes: 6 per week
                \item Weekend classes: 4 per week
            \end{itemize}
            Jennifer cannot take a combination of morning,
            night, and weekend classes. How many different classes
            does she have to choose from?
        \end{question}
        
        \begin{hint}{Hint}
            When questions are phrased as ``can choose \textbf{either ... OR ...}",
            this usually points to \textbf{adding} the size of both sets.
        \end{hint}

    
    \begin{intro}{The rule of sums with overlap}
        If the list to count can be split into two pieces of size $z$
        and $y$, and the pieces have $z$ objects in common, then the original
        list has $x + y - z$ entries. In terms of sets, we can write this as
        $n(A \cup B) = n(A) + n(B) - n(A \cap B)$ for all sets $A$ and $B$.
    \footnote{From Discrete Math Mathematical Reasoning and Proofs with Puzzles, Patterns and Games, by Ensley and Crawley}
    \end{intro}

        % - QUESTION --------------------------------------------------%
        \begin{question}{3}{10\%}
            Ryan has 107 games in his Steam library.
            Of those, 32 have the category ``action",
            14 have the category ``RPG",
            and 8 are categorized as both ``action" AND ``RPG", so they end up getting double-counted.
        \end{question}

        \begin{enumerate}
            \item[a.] How many games are action or RPG? (Use the rule of sums with overlap)
            \item[b.] How many games are NOT action or RPG? (The total, minus \textit{a.})
        \end{enumerate}
    
    \newpage
    %------------------------------------------------------------------%
    \section*{2. The rule of products}

        \begin{intro}{The rule of products}
            In combinatorics, the rule of product or multiplication principle is a
            basic counting principle (a.k.a. the fundamental principle of counting).
            Stated simply, it is the idea that if there are $a$ ways of doing something and $b$
            ways of doing another thing, then there are $a \cdot b$ ways of performing both actions.
            \footnote{From https://en.wikipedia.org/wiki/Rule\_of\_product}
        \end{intro}

        % - QUESTION --------------------------------------------------%
        \begin{question}{4}{10\%}
            Cristine is painting her nails, and can choose a color and a top coat.
            For colors, she has \textit{Red and Purple},
            and for top coats, she has \textit{Holo, Matte, and Gloss}.
            She will choose one color and one top coat.
        \end{question}

        \begin{enumerate}
            \item[a.] List out all possible combinations of color+top coat
            \item[b.] How many total combinations are there?
        \end{enumerate}

        \begin{hint}{Hint}
            When questions are phrased as ``choose ... AND ...",
            you will usually \textbf{multiply} the size of both setes.
        \end{hint}
        
        \hrulefill
        
        % - QUESTION --------------------------------------------------%
        \begin{question}{5}{10\%}
            On an old arcade machine, the highscore entries allow for
            3 letters for a player to sign their name.
        \end{question}

        \begin{enumerate}
            \item[a.] How many options are there for the first, second, and third letter?
            \item[b.] If you're choosing a first letter \textbf{and} a second letter \textbf{and} a third letter,
                how many possible combinations are there?
        \end{enumerate}

    \hrulefill

    \newpage
    %------------------------------------------------------------------%
    \section*{3. More questions?!}

        % - QUESTION --------------------------------------------------%
        \begin{question}{6}{15\%}
            In C++, a variable name can have any letters, with lowercase and uppercase
            counting as different characters. The variable name can also have
            numbers 0 through 9, except a number cannot be the first character of the name.
            \footnote{Based on a similar problem from Jim Van Horn's POGIL worksheet}
        \end{question}

        \begin{enumerate}
            \item[a.] For a single-character variable name, how many options are there? (Remember: can't use a number, but can use an uppercase or a lowercase letter).
            ~\\~\\
            \begin{tabular}{c}
                1. \\
                \fitb[2cm] \\
                \texttt{\lbrack a-zA-Z\rbrack}
            \end{tabular}
            
            \item[b.] For a two-character variable name, how many options are there? (Counting uppercase, lowercase, and numbers.)
            ~\\~\\
            \begin{tabular}{c c}
                1. & 2. \\
                \fitb[2cm] & \fitb[2cm] \\
                \texttt{\lbrack a-zA-Z\rbrack} & \texttt{\lbrack a-zA-Z0-9\rbrack}
            \end{tabular}
        \end{enumerate}


        % - QUESTION --------------------------------------------------%
        \begin{question}{7}{15\%}
            Suppose that we are rolling two dice...
            \footnote{Based on a similar problem from Jim Van Horn's POGIL worksheet}
        \end{question}

        \begin{enumerate}
            \item[a.] How many possible outcomes are there for rolling two 6-sided dice?
            \item[b.] How many outcomes are there where you get the same number twice?
            \item[c.] How many ways can you roll two dice and NOT get the same number on both?
        \end{enumerate}

        \newpage
        % - QUESTION --------------------------------------------------%
        \begin{question}{8}{20\%}
            Moneybank requires users to create a password. The password
            restrictions are that it must be 2 to 4 characters long, and
            each character can be either an uppercase letter \texttt{\lbrack A-Z\rbrack}
            or a digit \texttt{\lbrack 0-9\rbrack}. Answer the following questions
            to figure out how many possible passwords there are.
            \footnote{Based on a similar problem from Jim Van Horn's POGIL worksheet}
        \end{question}

        \begin{enumerate}
            \item[a.] How many possibilities are there for a 2-character password?
                (Take into account letters AND numbers!)
        \end{enumerate}
        ~\\ An additional restriction: Each password must contain at least one digit.
        Subtract all possible ways to have all six characters be a letter... Continue
        answering the questions to solve.
        
        \begin{enumerate}
            \item[b.] How many possibilities with just 26 letters?
            \item[c.] How many six-character passwords? (Take \textbf{a.} and subtract \textbf{b.})
        \end{enumerate}
        ~\\ Now follow the above to figure out for...

        \begin{enumerate}
            \item[d.] How many possible passwords with 3 characters?
            \item[e.] How many possible passwords with 4 characters?
            \item[f.] How many possible passwords, with either 2 characters, 3 characters, or 4 characters.
                (Are we using the rule of sums or the rule of products here? Notice the wording...)
        \end{enumerate}
        
    {
        %------------------------------------------------------------------%
        %- WORKSHEET ------------------------------------------------------%
        %------------------------------------------------------------------%
        \newpage
        \begin{center} \section*{Chapter \laChapter In-class Exercise Worksheet} \end{center}

        \iftoggle{answerkey}{
          \begin{answer} \begin{center} ANSWER KEY \end{center} \end{answer}
        }{}

        \paragraph{Team:}
        Please write down all people in your team. ~\\

        % table %
        \begin{tabular}{ p{6cm} p{6cm} }
            1. & 2. \\
            3. & 4.
        \end{tabular}
        % table %
        ~\\

        \hrulefill
        \subsection*{Grading}
                
        \begin{center}
            
            \begin{tabular}{ | l | l | l | l | }
                \hline
                \textbf{ Question } & \textbf{ Weight } & \textbf{ 0-4 } & \textbf{ Adjusted score }
                \\ \hline
                
                1 & 10\% & &    \\ \hline
                
                2 & 10\% & &    \\ \hline
                
                3 & 10\% & &    \\ \hline
                
                4 & 10\% & &    \\ \hline
                
                5 & 10\% & &    \\ \hline
                
                6 & 15\% & &    \\ \hline
                
                7 & 15\% & &    \\ \hline
                
                8 & 20\% & &    \\ \hline
                
                
            \end{tabular}
        \end{center}
    }

    \hrulefill
    %------------------------------------------------------------------%
 
    \section*{Answer sheet}
    % \iftoggle{answerkey}{ \begin{answer} asdfasdf \end{answer} }{ { ~\\ \raisebox{0pt}[2cm][0pt]{  } } }
    % \iftoggle{answerkey}{ \begin{answer} TRUE \end{answer} }{}

\begin{answersheetquestion}{1}{Books}{10}
\iftoggle{answerkey}{ \begin{answer} 3 + 4 = 7 books \end{answer} }{}
\end{answersheetquestion}

~\\

\hrulefill

~\\
\begin{answersheetquestion}{2}{Classes}{10}
\iftoggle{answerkey}{ \begin{answer} 3 + 6 + 4 = 13 \end{answer} }{}
\end{answersheetquestion}

~\\

\hrulefill

~\\
\begin{answersheetquestion}{3}{Video games}{10}

\end{answersheetquestion}

\begin{tabular}{p{6cm} p{6cm}}
    a. \iftoggle{answerkey}{ \begin{answer} $32 + 14 - 8 = 38$ \end{answer} }{}
    &
    b. \iftoggle{answerkey}{ \begin{answer} $107 - 69$ \end{answer} }{}
\end{tabular}



\newpage
\begin{answersheetquestion}{4}{Nail polish}{10}

    
\begin{tabular}{p{6cm} p{6cm}}
    a. \iftoggle{answerkey}{ \begin{answer} \{RH, RM, RG, PH, PM, PG\} \end{answer} }{}
    &
    b. \iftoggle{answerkey}{ \begin{answer} $2 \times 3 = 6$ \end{answer} }{}
\end{tabular}


\end{answersheetquestion}
    
\hrulefill

~\\
\begin{answersheetquestion}{5}{Highscore}{10}
\iftoggle{answerkey}{ \begin{answer}  \end{answer} }{}
\end{answersheetquestion}


\begin{tabular}{p{6cm} p{6cm}}
    a. \iftoggle{answerkey}{ \begin{answer} 26, 26, and 26 \end{answer} }{}
    &
    b. \iftoggle{answerkey}{ \begin{answer} $26 \times 26 \times 26 = 17,576$ \end{answer} }{}
\end{tabular}

\hrulefill

~\\
\begin{answersheetquestion}{6}{Variables}{15}
\iftoggle{answerkey}{ \begin{answer}  \end{answer} }{}
\end{answersheetquestion}


\begin{tabular}{p{6cm} p{6cm}}
    a. \iftoggle{answerkey}{ \begin{answer} $52$ \end{answer} }{}
    &
    b. \iftoggle{answerkey}{ \begin{answer} $(52) \times (62) = 3,224$ \end{answer} }{}
\end{tabular}


\hrulefill

~\\
\begin{answersheetquestion}{7}{Dice}{15}
\iftoggle{answerkey}{ \begin{answer}  \end{answer} }{}
\end{answersheetquestion}

\begin{enumerate}
    \item[a.] \iftoggle{answerkey}{ \begin{answer} $6 \times 6 = 36$ \end{answer} }{}
    \item[b.] \iftoggle{answerkey}{ \begin{answer} $6$ -- \{11, 22, 33, 44, 55, 66\}. \end{answer} }{}
    \item[c.] \iftoggle{answerkey}{ \begin{answer} $36 - 6 = 30$ \end{answer} }{}
\end{enumerate}

\hrulefill

~\\
\begin{answersheetquestion}{8}{Passwords}{20}
\iftoggle{answerkey}{ \begin{answer}  \end{answer} }{}
\end{answersheetquestion}

\begin{enumerate}
    \item[a.] \iftoggle{answerkey}{ \begin{answer} $(26+10)^{2} = 1,296$ \end{answer} }{}
    \item[b.] \iftoggle{answerkey}{ \begin{answer} $26^{2} = 676$ \end{answer} }{}
    \item[c.] \iftoggle{answerkey}{ \begin{answer} $1,296 - 676 = 620; or: (26+10)^{2} - (26)^{2}$ \end{answer} }{}
    \item[d.] \iftoggle{answerkey}{ \begin{answer} $(26+10)^{3} - (26)^{3} = 29,080$ \end{answer} }{}
    \item[e.] \iftoggle{answerkey}{ \begin{answer} $(26+10)^{4} - (26)^{4} = 1,222,640$ \end{answer} }{}
    \item[f.] \iftoggle{answerkey}{ \begin{answer} $620 + 29,080 + 1,222,640 = 1,252,340$ \end{answer} }{}
\end{enumerate}
    
\end{document}
