\documentclass[a4paper,12pt]{book}
\usepackage[utf8]{inputenc}
\title{}
\author{Rachel Morris}
\date{\today}

\usepackage{rachwidgets}
\usepackage{fancyhdr}
\usepackage{lastpage}
\usepackage{dirtree}
\usepackage{boxedminipage}

\setcounter{chapter}{6}
\setcounter{section}{1}
\newcommand{\laChapter}{6.2 Introduction\ }

\newcommand{\laClass}{CS 211\ }
\newcommand{\laSemester}{Fall 2017\ }
\newcounter{question}

\pagestyle{fancy}
\fancyhf{}
\lhead{CS 211 Exercise}
\chead{Fall 2017}
\rhead{Ch \laChapter}
\rfoot{\thepage\ of \pageref{LastPage}}
\lfoot{\scriptsize Compiled by Rachel Morris, last updated \today}

\renewcommand{\headrulewidth}{2pt}
\renewcommand{\footrulewidth}{1pt}

\begin{document}

    %\toggletrue{answerkey}
    \togglefalse{answerkey}


    \notonkey{
    %- Team Info ------------------------------------------------------%

    \paragraph{Team name:}

    ~\\~\\
    Please write down all people in your team. ~\\

    % table %
    \begin{tabular}{ p{6cm} p{6cm} }
        1. & 2. \\ \\
        3. & 4.
    \end{tabular}
    % table %
    ~\\

    \hrulefill
    \subsection*{Grading}

    \begin{center}

        \begin{tabular}{ | l | l | l | }
            \hline
            \textbf{ Question } & \textbf{ Score } & \textbf{ Max } 
            \\ \hline
            1 &  & 3     \\ \hline
            2 &  & 3     \\ \hline
            3 &  & 1     \\ \hline
            4 &  & 2     \\ \hline
            5 &  & 2     \\ \hline
            6 &  & 1     \\ \hline
            7 &  & 1     \\ \hline
            & &  \\ \hline
            Total & & 13
            \\ \hline
        \end{tabular}
    \end{center}
    }{}

\notonkey{ \newpage }{ \hrulefill }

    %------------------------------------------------------------------%
    \section{The Sum Rule}

    \notonkey{
        \begin{intro}{Disjoint events}
            Two events are said to be \textbf{disjoint} (or \textit{mutually exclusive}
            if they cannot occur simultaneously.
            \footnote{From Discrete Math by Ensley and Crawley, page 448}
        \end{intro}
    }{}

        % -------------------------------------------------------------%
        % - QUESTION --------------------------------------------------%
        % -------------------------------------------------------------%
        \stepcounter{question}
        \begin{question}{\thequestion}{3}
        	% Practice problem 1
        	For each of the experiments given below, decide if the
				events described are disjoint:

				\begin{itemize}
					\item[a.]	When tossing a coin four times, let $E_1$ be the
					event that there are exactly three heads and $E_2$ be the event
					that there are exactly two heads. ~\\
					\solution{ When tossing a coin four times, no outcome can
					consist of ``exactly two heads'' and also ``exactly three heads'';
					hence, these two events are disjoint.}{ { ~\\ \raisebox{0pt}[2cm][0pt]{  } } }

					\item[b.]	When choosing four cards, let $E_1$ be the event that
					the cards have the same value and $E_2$ be the event that the cards
					have the same suit. ~\\
					\solution{When choosing four cards, if the four cards all have
					the same value, then they cannot all have the same suit; hence, these
					two events are disjoint.}{ { ~\\ \raisebox{0pt}[2cm][0pt]{  } } }

					\item[c.] When choosing a committee of three people from a
					club with 8 men and 12 women, let $E_1$ be the event that the committee
					has a woman and let $E_2$ be the event that the committee has a man. ~\\
					\solution{When choosing a committee of three people
					from a club with 8 men and 12 women, there are many ways in which
					the committee can include a woman and a man,
					so these events are not disjoint.}{ { ~\\ \raisebox{0pt}[2cm][0pt]{  } } }
				\end{itemize}
        \end{question}

\notonkey{ \newpage }{ \hrulefill }

    \notonkey{
        \begin{intro}{The Sum Rule, Theorem 1}
            If $E_1$ and $E_2$ are disjoint events in a given experiment,
            then the probability that $E_1$ or $E_2$ occurs is the sum of
            $Prob(E_1)$ and $Prob(E_2)$. That is,

            $$Prob(E_1 or E_2) = Prob(E_1) + Prob(E_2)$$

            for disjoint events.
            \footnote{From Discrete Math by Ensley and Crawley, page 449}
        \end{intro}
    }{}

        % -------------------------------------------------------------%
        % - QUESTION --------------------------------------------------%
        % -------------------------------------------------------------%
        \stepcounter{question}
        \begin{question}{\thequestion}{3}
            % Exercise 2
            A card is drawn from an ordinary deck of 52 cards. Show how
            to use the basic sum rule to find the probability that
            the card is...

            \begin{itemize}
                \item[a.] An ace or a jack.

                    \notonkey{
                        \begin{hint}{Hint}
                            $E_{1}$ is the set of outcomes where you get an Ace... \\
                            \{ Ace-Heart, Ace-Diamond, Ace-Spade, Ace-Club \}
                            and the probability of getting an Ace, $Prob(E_{1})$
                            is 4 outcomes out of 52, or $\frac{1}{13}$.
                        \end{hint}
                    }{}
                    \solution{
                        $$Prob(E_{1} or E_{2}) = Prob(E_{1}) + Prob(E_{2}) = \frac{1}{13} + \frac{1}{13} = \frac{2}{13}$$
                    }{ { ~\\ \raisebox{0pt}[2cm][0pt]{  } } }

                \item[b.] A diamond or a black jack, queen, or king card. ~\\
                    \solution{
                    $E_{1}$ is the outcomes where you get a diamond,
                    and $E_{2}$ is the outcomes where you get a black Jack, King, or Queen.
                    These sets are disjoint, so...
                    $$Prob(E_{1} or E_{2}) = Prob(E_{1}) + Prob(E_{2}) = \frac{1}{4} + \frac{6}{52} = \frac{19}{52}$$
                    }{ { ~\\ \raisebox{0pt}[2cm][0pt]{  } } }

                \item[c.] An even number value or a red jack, queen, or king card. ~\\
                    \solution{
                    $E_{1}$ is the outcomes where the card has an even numbered value,
                    and $E_{2}$ is the set of outcomes with a red Jack, King, or Queen.
                    These sets are disjoint, so...
                    $$Prob(E_{1} or E_{2}) = Prob(E_{1} + Prob(E_{2}) = \frac{5}{13} + \frac{6}{52} = \frac{1}{2}$$
                    }{ { ~\\ \raisebox{0pt}[2cm][0pt]{  } } }
            \end{itemize}
        \end{question}

\notonkey{ \newpage }{ \hrulefill }

    \notonkey{
        \begin{intro}{The General Sum Rule, Theorem 2}
            If $E_1$ and $E_2$ are any events in a given experiment,
            then the probability that $E_1$ or $E_2$ occurs is given by
            $$ Prob(E_1 or E_2) = Prob(E_1) + Prob(E_2) - Prob(E_1 and E_2) $$

            If $E_1$ and $E_2$ are disjoint, then $E_1 \cap E_2 = \emptyset$,
            so $Prob(E_1 and E_2) = 0$.
            \footnote{From Discrete Math by Ensley and Crawley, page 450}
        \end{intro}
    }{}

        % -------------------------------------------------------------%
        % - QUESTION --------------------------------------------------%
        % -------------------------------------------------------------%
        \stepcounter{question}
        \begin{question}{\thequestion}{1}
            % Exercise 7
            What is the probability that when a pair of dice are rolled,
            either (at least) one die shows a 5 or the dice sum to 8?

            \textbf{What is the sample size?} \tab $S = $ \solution{36}{ \fitb }

            \begin{center}
            	\begin{tabular}{c c c c c c}
            		(1,1) & (1,2) & (1,3) & (1,4) & (1,5) & (1,6) \\
						(2,1) & (2,2) & (2,3) & (2,4) & (2,5) & (2,6) \\
            		(3,1) & (3,2) & (3,3) & (3,4) & (3,5) & (3,6) \\
						(4,1) & (4,2) & (4,3) & (4,4) & (4,5) & (4,6) \\
            		(5,1) & (5,2) & (5,3) & (5,4) & (5,5) & (5,6) \\
						(6,1) & (6,2) & (6,3) & (6,4) & (6,5) & (6,6)
            	\end{tabular}
            \end{center}


            \textbf{How many outcomes $n(E_1)$ are there where you get either 5 for the first die, 5 for the second die, or 5 for both dice? Write out the set of $E_1$.}
            ~\\
            \solution{
            	$E_1 = \{ (5, 1), (5, 2), (5, 3), (5, 4), (5, 5), (5, 6),
            		(6, 5), (4, 5), (3, 5), (2, 5), (1, 5) \}
            		\\ n(E_1) = 11$
            }{~\\}

				\textbf{How many outcomes $n(E_2)$ are there where the two dice sum to 8? Write out the set of $E_2$.} ~\\
				\solution{
					$E_2 = \{ (2, 6), (6, 2), (3, 5), (5, 3), (4, 4) \} \\
					n(E_2) = 5$
				}
				{~\\}

				\textbf{What is the amount of overlap $n(E_1 \tab[0.5cm]AND\tab[0.5cm] E_2)$? Write out this set.} ~\\
				\solution{
					$n(E_1 \tab[0.5cm]AND\tab[0.5cm] E_2) = 2$
				}{~\\}

				\textbf{Use the General Sum Rule to find the probability that you will get either
					at least one die showing a 5, OR the dice sum to 8.}

            \solution{$\frac{11}{36} + \frac{5}{36} - \frac{2}{36} = \frac{7}{18}$}{~\\~\\}
        \end{question}

\notonkey{ \newpage }{ \hrulefill }

    %------------------------------------------------------------------%
    \section{The Product Rule}

    \notonkey{
        \begin{intro}{Independent events}
            Two events are said to be \textbf{independent} if the occurrence
            of one event is not influenced by the occurred (or nonoccurrence)
            of the other event.
            \footnote{From Discrete Math by Ensley and Crawley, page 451}
        \end{intro}
    }{}

        % -------------------------------------------------------------%
        % - QUESTION --------------------------------------------------%
        % -------------------------------------------------------------%
        \stepcounter{question}
        \begin{question}{\thequestion}{2}
            % Practice Problem 3
            For each of the following experiments given below,
            decide if the events described are independent:

            \begin{itemize}
                \item[a.] When rolling a 6-sided die four times, let $E_{1}$
                    be the event that the first two rolls sum to 7 and let $E_{2}$
                    be the event that the last two rolls sum to 10. ~\\
                    \solution{The events are independent. The first two rolls
                    have no influence on the last two rolls.}{ { ~\\ \raisebox{0pt}[4cm][0pt]{  } } }

                \item[b.] When choosing a committee of three dogs from a club of
                    8 corgis and 12 labradors, let $E_{1}$ be the event that
                    the committee has a labrador and let $E_{2}$ be the
                    event that the committee has a corgi. ~\\
                    \solution{The events are not independent. The probability
                    of the committee having a corgi ($E_{2}$) is different
                    when $E_{1}$ occurs than it is when $E_{1}$ does not occur.
                    Specifically, if $E_{1}$ does not occur, then $E_{2}$ happens
                    for sure (i.e., its probability is 1), and if $E_{1}$ does occur,
                    then $E_{2}$ is not guaranteed to happen (i.e., probability is
                    less than 1).
                    }{ ~\\ }

            \end{itemize}
        \end{question}

\notonkey{ \newpage }{ \hrulefill }

    \notonkey{
        \begin{intro}{The Product Rule, Theorem 3}
            If $E_{1}$ and $E_{2}$ are independent events in a given
            experiment, then the probability that both $E_{1}$ and $E_{2}$
            occur is the product of $Prob(E_{1}$ and $Prob(E_{2}$. That is,
            $$ Prob(E_{1} and E_{2}) = Prob(E_{1}) \cdot Prob(E_{2}) $$
            for independent events.
            \footnote{From Discrete Math by Ensley and Crawley, page 452}
        \end{intro}
    }{}
    
        % -------------------------------------------------------------%
        % - QUESTION --------------------------------------------------%
        % -------------------------------------------------------------%
        \stepcounter{question}
        \begin{question}{\thequestion}{2}
            % Example 6
            Suppose I have a ``loaded" die for which the probability of a 6
            appearing is $\frac{1}{2}$, while the probability of each of the other faces
            appearing is $\frac{1}{10}$. What is the probability of getting a 5
            and then a 6 on two tosses of the loaded die?

            ~\\
            First identify $E_{1}$ and $E_{2}$. These events are independent,
            so you can use the Product Rule to find $Prob(E_{1} and E_{2})$.

            \solution{
            The events $E_{1}$, ``getting a 5 on the first toss",
            and $E_{2}$, ``getting a 6 on the second toss", are independent,
            so by the product rule,
            $$Prob(E_{1} and E_{2}) = Prob(E_{1}) \cdot Prob(E_{2}) = \frac{1}{10} \cdot \frac{1}{2}
            = \frac{1}{20}$$
            }{ { ~\\ \raisebox{0pt}[6cm][0pt]{  } } }
        \end{question}

    \notonkey{
        \begin{intro}{The probability of $E_{1}$ given $E_{2}$}
            Given events $E_{1}$ and $E_{2}$ for some experiment, we
            define the probability of $E_{1}$ given $E_{2}$, denoted by
            $Prob(E_{1} | E_{2})$, as the probability that $E_{1}$ happens
            given that $E_{2}$ occurs. Note that if $E_{1}$ and $E_{2}$
            are independent, then $Prob( E_{1} | E_{2}) = Prob(E_{1})$.
            \footnote{From Discrete Math by Ensley and Crawley, page 452}
        \end{intro}
    }{}
    
\notonkey{ \newpage }{ \hrulefill }

    \notonkey{
        \begin{intro}{The General Product Rule, Theorem 4}
            If $E_{1}$ and $E_{2}$ are any events in a given experiment, then
            the probability that both $E_{1}$ and $E_{2}$ occur is given by

            $$Prob(E_{1} and E_{2}) = Prob(E_{2}) \cdot Prob(E_{1} | E_{2}) $$
            $$ = Prob(E_{1}) \cdot Prob(E_{2} | E_{1}) $$

            Note that if $E_{1}$ and $E_{2}$ are independent, then this says
            the same thing as Theorem 3.
            \footnote{From Discrete Math by Ensley and Crawley, page 453}
        \end{intro}
    }{}
    
        % -------------------------------------------------------------%
        % - QUESTION --------------------------------------------------%
        % -------------------------------------------------------------%
        \stepcounter{question}
        \begin{question}{\thequestion}{1}
            % Example 8
            Two marbles are chosen from a bag containing three red, five white, and eight green marbles,
            so there are 16 total marbles.
            What is the probability that both are red?

            ~\\
            Here, the event $R_{1}$ is ``the first marble is red",
            and the event $R_{2}$ is ``the second marble is red".

            ~\\
            What is $Prob(R_{1})$?
            \solution{ $\frac{3}{16}$ }{ ~\\ \raisebox{0pt}[1cm][0pt]{  } }

            ~\\
            Since $R_{2}$ depends on $R_{1}$ occurring, after $R_{1}$
            occurs, there are 15 marbles left. One red marble
            has been selected, so there are 2 red marbles left.

            ~\\
            What is $Prob(R_{2} | R_{1})$?
            \solution{ $\frac{2}{15}$ }{ ~\\ \raisebox{0pt}[1cm][0pt]{  } }

            ~\\
            With this information, what is $Prob(R_{1} and R_{2})$?

            \solution{
                $Prob(R_{1}) = \frac{3}{16}$, $Prob(R_{2}) = \frac{2}{15}$. \\
                $Prob(R_{1} and R_{2}) = Prob(R_{1}) \cdot Prob(R_{2} | R_{1})$ \\
                $ = \frac{6}{240} = \frac{1}{40}$
                }{}

        \end{question}

\notonkey{ \newpage }{ \hrulefill }

        % -------------------------------------------------------------%
        % - QUESTION --------------------------------------------------%
        % -------------------------------------------------------------%
        \stepcounter{question}
        \begin{question}{\thequestion}{1}
            % Example 8
            Two marbles are chosen from a bag containing three red, five white, and eight green marbles,
            so there are 16 total marbles.
            What is the probability that one is white and one is green?

            ~\\
            Let's say we have the events
            $W_{1}$ (White first), $W_{2}$ (White second), $G_{1}$ (Green first), and
            $G_{2}$ (Green second), so we can get our result in two ways: with
            $(W_{1}, G_{2})$ \textbf{OR} with$(G_{1}, W_{2})$, so you can calculate
            the result as

            $$Prob(W_{1} and G_{2}) + Prob(G_{1} and W_{2})$$

            \solution{
                $W_{1}$ will be white as the first, and $G_{2}$ will be green as the second.

                $Prob(W_{1} and G_{1} + Prob(G_{1} and W_{2}) = \\
                    Prob(W_{1}) \cdot Prob(G_{2} | W_{1}) + Prob(G_{1}) \cdot Prob(W_{2}|G_{1}) \\
                    = \frac{5}{16} \cdot \frac{8}{15} + \frac{8}{16} \cdot \frac{5}{15} = \frac{1}{3}$
                }{{ ~\\ \raisebox{0pt}[4cm][0pt]{  } }}
        \end{question}

\end{document}
