\documentclass[a4paper,12pt]{book}
\usepackage[utf8]{inputenc}
\title{}
\author{Rachel Morris}
\date{\today}

\usepackage{rachwidgets}
\usepackage{fancyhdr}
\usepackage{lastpage}
\usepackage{dirtree}
\usepackage{boxedminipage}

\setcounter{chapter}{6}
\setcounter{section}{4}
\newcommand{\laChapter}{6.5 Recursion Revisited\ }

\newcommand{\laClass}{CS 211\ }
\newcommand{\laSemester}{Fall 2017\ }
\newcounter{question}

\pagestyle{fancy}
\fancyhf{}
\lhead{CS 211 Exercise}
\chead{Fall 2017}
\rhead{Ch \laChapter}
\rfoot{\thepage\ of \pageref{LastPage}}
\lfoot{\scriptsize Compiled by Rachel Morris, last updated \today}

\renewcommand{\headrulewidth}{2pt}
\renewcommand{\footrulewidth}{1pt}

\begin{document}

    %\toggletrue{answerkey}
    \togglefalse{answerkey}

    %------------------------------------------------------------------%
    \section{Review}

        \begin{intro}{Complement}
            Given an event $E$,
            $$ Prob(E) + Prob(\bar{E}) = 1 $$
            Where $\bar{E}$ is the complement of the event $E$.

            \paragraph{Example question}
            What is the probability that for a six-sided die rolled
            three times the same result comes up more than once?

            \begin{enumerate}
                \item[a.] What is the sample space $S$? \\
                    \{1, 2, 3, 4, 5, 6\}

                \item[b.] What is the event $E$ (in English)? \\
                    The set of outcomes that use the same \# more than once.

                \item[c.] What is the complement of $\bar{E}$ (in English)? \\
                    The set of outcomes that are all different numbers.

                \item[d.] What \textit{structure type} is $\bar{E}$? What is $n$ and $r$? \\
                    Permutation, $n = 6$, $r = 3$

                \item[e.] Calculate $Prob(\bar{E})$ \\
                    $Prob(\bar{E}) = n(\bar{E}) / n(S) =$
                    $\frac{P(6,3)}{6^{3}} = \frac{5}{9} = 0.\bar{5}$

                \item[f.] Calculate the probability for the Event $Prob(E)$ using the proposition. \\
                    $1 - Prob(\bar{E}) = 1 - 0.55 \approx 0.44$
            \end{enumerate}

        \end{intro}

        \begin{intro}{Expected (average) value}

            For a given probability experiment, let $X$ be a random
            variable whose possible values come from the set of numbers
            $ x_{1}, ..., x_{n} $. Then the \textbf{expected value of $X$},
            denoted by $E[X]$, is the sum
            $$ E[X] = (x_{1}) \cdot Prob(X = x_{1}) + ... + (x_{n}) \cdot Prob(X = x_{n}) $$

        \end{intro}

        \newpage


    %------------------------------------------------------------------%
    \section{Recursion revisited}

        \begin{intro}{\ }
            \textbf{Average trials until getting first value} \\
            A common type of problem in this section is to find the
            average amount of trials run until you get some value for
            the first time... For example, rolling a die until you
            get a ``1" for the first time.

            ~\\
            Let's say there's some probability $p$ of success, and $X$
            is the amount of trials (rolls, flips, etc.) until the first
            value is received, and $E[X]$ is the average amount of
            trials that will run.

            ~\\
            We start with this formula:
            $$ E[X] = p(1) + (1 - p)(1 + E[X]) $$

            The probability should be known, so by simplifying you
            can solve for $E[X]$ to find the result.

            \paragraph{Example 1}
                Find the average number of tosses of a fair coin that it takes
                to get a result of heads for the first time.

                The probability of getting a heads is $p = (1/2)$, so we
                can write this out as:

                $$ E[X] = \frac{1}{2}(1) + (1 - \frac{1}{2})(1 + E[X]) $$

                and simplify...

                $$ E[X] = \frac{1}{2} + \frac{1}{2} + \frac{1}{2}(E[X]) $$

                $$ E[X] - \frac{1}{2}(E[X]) = 1 $$

                $$ \frac{1}{2}(E[X]) = 1 $$

                $$ E[X] = 2 $$
        \end{intro}

        \newpage
        % -------------------------------------------------------------%
        % - QUESTION --------------------------------------------------%
        % -------------------------------------------------------------%
        \stepcounter{question}
        \begin{question}{\thequestion}{2}
            % Exercise 1
            What is the expected number of rolls of a six-sided die
            that is rolled until a 1 appears?

            \solution{
                $ E[X] = \frac{1}{6}(1) + \frac{5}{6}(1 + E[X]) $ \\
                $ E[X] = 6 $
            }{ { ~\\ \raisebox{0pt}[4cm][0pt]{  } } }
        \end{question}

        \hrulefill

        % -------------------------------------------------------------%
        % - QUESTION --------------------------------------------------%
        % -------------------------------------------------------------%
        \stepcounter{question}
        \begin{question}{\thequestion}{2}
            % Exercise 3
            A pair of dice are thrown until at least one of the die comes
            up 1 for the first time. How many tosses,
            on average, are required?

            We are rolling two die, which comes out to:

            \begin{center}
                \begin{tabular}{c c c c c c}
                        \textbf{(1,1)} & \textbf{(1,2)} & \textbf{(1,3)} & \textbf{(1,4)} & \textbf{(1,5)} & \textbf{(1,6)} \\
                        \textbf{(2,1)} & (2,2) & (2,3) & (2,4) & (2,5) & (2,6) \\
                        \textbf{(3,1)} & (3,2) & (3,3) & (3,4) & (3,5) & (3,6) \\
                        \textbf{(4,1)} & (4,2) & (4,3) & (4,4) & (4,5) & (4,6) \\
                        \textbf{(5,1)} & (5,2) & (5,3) & (5,4) & (5,5) & (5,6) \\
                        \textbf{(6,1)} & (6,2) & (6,3) & (6,4) & (6,5) & (6,6)
                \end{tabular}
            \end{center}

            \begin{itemize}
                \item[a.]   What is the sample size?
                \solution{ 36 }{ \fitb }

                \item[b.]   How many of these rules have at least one 1?
                \solution{ 11 }{ \fitb }

                \item[c.]   What is the probability of getting at least one?
                \solution{ $\frac{11}{36}$ }{ \fitb } \\
                ($Prob(E) = n(E) / n(S)$)

                \item[d.]   Use the formula to find the expected value (average trials).
                    \solution{
                        $ E[X] = \frac{11}{36}(1) + \frac{25}{36}(1 + E[X]) $ \\
                        $ E[X] = \frac{36}{11} $
                    }{ { ~\\ \raisebox{0pt}[4cm][0pt]{  } } }

            \end{itemize}

        \end{question}

\end{document}
