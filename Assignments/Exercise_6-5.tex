\documentclass[a4paper,12pt]{book}
\usepackage[utf8]{inputenc}
\title{}
\author{Rachel Morris}
\date{\today}

\usepackage{rachwidgets}
\usepackage{fancyhdr}
\usepackage{lastpage}
\usepackage{dirtree}
\usepackage{boxedminipage}

\setcounter{chapter}{6}
\setcounter{section}{4}
\newcommand{\laChapter}{6.5 Recursion Revisited\ }

\newcommand{\laClass}{CS 211\ }
\newcommand{\laSemester}{Fall 2017\ }
\newcounter{question}

\pagestyle{fancy}
\fancyhf{}
\lhead{CS 211 Exercise}
\chead{Fall 2017}
\rhead{Ch \laChapter}
\rfoot{\thepage\ of \pageref{LastPage}}
\lfoot{\scriptsize Compiled by Rachel Morris, last updated \today}

\renewcommand{\headrulewidth}{2pt}
\renewcommand{\footrulewidth}{1pt}

\begin{document}

    %\toggletrue{answerkey}
    \togglefalse{answerkey}
    
    %------------------------------------------------------------------%
    \section{Review}

        \begin{intro}{Complement}
            Given an event $E$,
            $$ Prob(E) + Prob(\bar{E}) = 1 $$
            Where $\bar{E}$ is the complement of the event $E$.
                
            \paragraph{Example question}        
            What is the probability that for a six-sided die rolled
            three times the same result comes up more than once?
            
            \begin{enumerate}
                \item[a.] What is the sample space $S$? \\
                    \{1, 2, 3, 4, 5, 6\}
                
                \item[b.] What is the event $E$ (in English)? \\
                    The set of outcomes that use the same \# more than once.
                    
                \item[c.] What is the complement of $\bar{E}$ (in English)? \\
                    The set of outcomes that are all different numbers. 
                
                \item[d.] What \textit{structure type} is $\bar{E}$? What is $n$ and $r$? \\
                    Permutation, $n = 6$, $r = 3$
                
                \item[e.] Calculate $Prob(\bar{E})$ \\
                    $Prob(\bar{E}) = n(\bar{E}) / n(S) =$
                    $\frac{P(6,3)}{6^{3}} = \frac{5}{9} = 0.\bar{5}$
                
                \item[f.] Calculate the probability for the Event $Prob(E)$ using the proposition. \\
                    $1 - Prob(\bar{E}) = 1 - 0.55 \approx 0.44$
            \end{enumerate}

        \end{intro}

        \begin{intro}{Expected (average) value}
            
            For a given probability experiment, let $X$ be a random
            variable whose possible values come from the set of numbers
            $ x_{1}, ..., x_{n} $. Then the \textbf{expected value of $X$},
            denoted by $E[X]$, is the sum
            $$ E[X] = (x_{1}) \cdot Prob(X = x_{1}) + ... + (x_{n}) \cdot Prob(X = x_{n}) $$

        \end{intro}

        \newpage

        
    %------------------------------------------------------------------%
    \section{Recursion revisited}

        \begin{intro}{\ }
            \textbf{Average trials until getting first value} \\
            A common type of problem in this section is to find the
            average amount of trials run until you get some value for
            the first time... For example, rolling a die until you
            get a ``1" for the first time.

            ~\\
            Let's say there's some probability $p$ of success, and $X$
            is the amount of trials (rolls, flips, etc.) until the first
            value is received, and $E[X]$ is the average amount of
            trials that will run.

            ~\\
            We start with this formula:
            $$ E[X] = p(1) + (1 - p)(1 + E[X]) $$

            The probability should be known, so by simplifying you
            can solve for $E[X]$ to find the result.

            \paragraph{Example 1}
                Find the average number of tosses of a fair coin that it takes
                to get a result of heads for the first time.

                The probability of getting a heads is $p = (1/2)$, so we
                can write this out as:

                $$ E[X] = \frac{1}{2}(1) + (1 - \frac{1}{2})(1 + E[X]) $$

                and simplify...

                $$ E[X] = \frac{1}{2} + \frac{1}{2} + \frac{1}{2}(E[X]) $$

                $$ E[X] - \frac{1}{2}(E[X]) = 1 $$

                $$ \frac{1}{2}(E[X]) = 1 $$

                $$ E[X] = 2 $$
        \end{intro}

        \newpage
        % -------------------------------------------------------------%
        % - QUESTION --------------------------------------------------%
        % -------------------------------------------------------------%
        \stepcounter{question}
        \begin{question}{\thequestion}{2}
            % Exercise 1
            What is the expected number of rolls of a six-sided die
            that is rolled until a 1 appears?

            \solution{
                $ E[X] = \frac{1}{6}(1) + \frac{5}{6}(1 + E[X]) $ \\
                $ E[X] = 6 $
            }{ { ~\\ \raisebox{0pt}[4cm][0pt]{  } } }
        \end{question}

        \hrulefill

        % -------------------------------------------------------------%
        % - QUESTION --------------------------------------------------%
        % -------------------------------------------------------------%
        \stepcounter{question}
        \begin{question}{\thequestion}{2}
            % Exercise 3
            A pair of dice are thrown until at least one of the die comes
            up 1 for the first time. How many tosses,
            on average, are required?

            We are rolling two die, which comes out to:
            
            \begin{center}
                \begin{tabular}{c c c c c c}
                        \textbf{(1,1)} & \textbf{(1,2)} & \textbf{(1,3)} & \textbf{(1,4)} & \textbf{(1,5)} & \textbf{(1,6)} \\
                        \textbf{(2,1)} & (2,2) & (2,3) & (2,4) & (2,5) & (2,6) \\
                        \textbf{(3,1)} & (3,2) & (3,3) & (3,4) & (3,5) & (3,6) \\
                        \textbf{(4,1)} & (4,2) & (4,3) & (4,4) & (4,5) & (4,6) \\
                        \textbf{(5,1)} & (5,2) & (5,3) & (5,4) & (5,5) & (5,6) \\
                        \textbf{(6,1)} & (6,2) & (6,3) & (6,4) & (6,5) & (6,6)
                \end{tabular}
            \end{center}

            \begin{itemize}
                \item[a.]   What is the sample size?
                \solution{ 36 }{ \fitb }

                \item[b.]   How many of these rules have at least one 1?
                \solution{ 11 }{ \fitb }

                \item[c.]   What is the probability of getting at least one?
                \solution{ $\frac{11}{36}$ }{ \fitb } \\
                ($Prob(E) = n(E) / n(S)$)

                \item[d.]   Use the formula to find the expected value (average trials).
                    \solution{
                        $ E[X] = \frac{1}{6}(1) + \frac{5}{6}(1 + E[X]) $ \\
                        $ E[X] = 6 $
                    }{ { ~\\ \raisebox{0pt}[4cm][0pt]{  } } }

            \end{itemize}

        \end{question}

    \newpage

    \subsection{Running trials in code}

        Let's write a recursive function to run these types of trials
        for us and get a simulated average value.

        ~\\
        In Visual Studio create a new \textbf{Empty Project}, and add
        one source file to it. Make sure to include the following libraries:

\begin{verbatim}
#include <iostream>     // output
#include <cstdlib>      // random
#include <ctime>        // time
using namespace std;    // standard library
\end{verbatim}

        Start with the following shell function:
        
\begin{verbatim}
int RunTrial( int min, int max, int stopAtFirst, int trialCount )
{
}
\end{verbatim}

        And main():

\begin{verbatim}
int main()
{
    srand( time( NULL ) );
    int min, max, stopAtFirst, count;
    
    return 0;
}
\end{verbatim}

\newpage

\textbf{Starter code}

\begin{lstlisting}[style=code]
#include <iostream>     // output
#include <cstdlib>      // random
#include <ctime>        // time
using namespace std;    // standard library

int RunTrial(   int min, int max,
                int stopAtFirst, int trialCount )
{
}

int main()
{
    // seed the random # generator
    srand( time( NULL ) );  
    int min, max, stopAtFirst, count;
    
    return 0;
}
\end{lstlisting}
    ~\\
    
    In C++, you can display text to the screen with a \texttt{cout} statement...

\begin{verbatim}
cout << "Message";
\end{verbatim}

    And you can get input with a \texttt{cin} statement, storing it in a variable...

\begin{verbatim}
cin >> min;
\end{verbatim}

    \textbf{Get user input:}
    Ask the user to enter the \textbf{Minimum value}, \textbf{Maximum value},
    \textbf{Which value to stop at}, and \textbf{How many times to run it}.
    Store these in the variables declared at the top of main().

    ~\\
    When running the program, the user can enter the following to
    simulate a die roll or a coin flip:

    \begin{center}
        \begin{tabular}{ l c c }
            & min & max
            \\ \hline
            Die & 1 & 6
            \\
            Coin & 1 & 2
        \end{tabular}
    \end{center}

    \newpage


    \textbf{Total trials:}
    Create an integer variable that will store
    the total amount of trials that were run across all experiments.
    Make sure to initialize it to 0.
    ~\\
    
    \textbf{Experiment run:}
    After getting the user's input, you will need to write a for-loop
    to run the experiment the specified amount of times.
    ~\\

    \textbf{Calling the function:}
    Within the for-loop, you will call the \texttt{RunTrial} function.
    The output of this function is the total amount of trials that
    were ran before getting the specified value for the first time.
    You will add this value onto the \textbf{totalTrials} variable.
    ~\\
    
    \textbf{Calculating the average:}
    Once the experiments has completed (the for-loop is over), you will
    calculate the average amount of trials that were ran.
    Create a float variable called \textbf{average} and divide
    the \textbf{totalTrials} by the \textbf{count}.
    Make sure you're doing float division or it won't turn out correctly.

    ~\\

    See the next page for the main() code so far.


\newpage
\begin{lstlisting}[style=code]
int main()
{
    // seed the random # generator
    srand( time( NULL ) );  
    int min, max, stopAtFirst, count;
    
    cout << "Minimum value: ";
    cin >> min;

    cout << "Maximum value: ";
    cin >> max;

    cout << "Stop at what: ";
    cin >> stopAtFirst;
    
    cout << "How many times to run: ";
    cin >> count;

    int totalTrials = 0;

    for ( int i = 0; i < count; i++ )
    {
        cout << endl;
        cout << "Running trial set " << i << "... \t";
        int trials = RunTrial( min, max,
            stopAtFirst, 1 );
        totalTrials += trials;
        cout << "\t" << trials << " trials ran";
    }
    
    float averageTrials =
        float( totalTrials ) / float( count );
        
    cout << "Average trials ran: " << averageTrials
        << endl << endl;
    
    return 0;
}
\end{lstlisting}

    \newpage

    Now to work on the recursive function.
    
\begin{verbatim}
int RunTrial( int min, int max, int stopAtFirst, int trialCount )
{
}
\end{verbatim}

    A \textbf{Recursive Function} is a function that will call itself
    until the job is done. It is useful for breaking a problem into pieces.

    At the top of this function, you will get a random value with the following:

\begin{verbatim}
// Get a random value between MIN and MAX, inclusive
int diff = (max + 1 - min);
int n = rand() % diff + min;
cout << n << " ";
\end{verbatim}

    Recursive functions need a \textbf{Base-case}, which is the scenario
    where it will end. In our case, the function stops once we get
    the \textbf{stopAtFirst} value.
    Use an if statement to check if the value of \textbf{n} is equal
    to \textbf{stopAtFirst}. If it is, return the current value of the
    \textbf{trialCount}.

    Otherwise, we will call the recursive function.
    All the variables are the same, except we add 1 to the trial count.
    
\begin{verbatim}
return RunTrial( min, max, stopAtFirst, trialCount + 1 );
\end{verbatim}

    Altogether, it will look like this:

\begin{lstlisting}[style=code]
int RunTrial( int min, int max,
    int stopAtFirst, int trialCount )
{
    int diff = (max + 1 - min);
    int n = rand() % diff + min;

    cout << n << " ";
    
    if ( n == stopAtFirst ) // Base case
    {
        return trialCount;
    }

    return RunTrial( min, max,
        stopAtFirst, trialCount + 1 );
}
\end{lstlisting}

\newpage

    When the program is run, we can execute the experiment a large
    amount of times to see if the average that comes out is close
    to the expected value we calculate.
    
\begin{lstlisting}[style=output]
Minimum value: 1
Maximum value: 2
Stop at what: 1
How many times to run: 10000

(...)

Running trial set 9989... 	1 	1 trials ran
Running trial set 9990... 	1 	1 trials ran
Running trial set 9991... 	1 	1 trials ran
Running trial set 9992... 	1 	1 trials ran
Running trial set 9993... 	1 	1 trials ran
Running trial set 9994... 	2 2 2 1 	4 trials ran
Running trial set 9995... 	2 1 	2 trials ran
Running trial set 9996... 	2 2 1 	3 trials ran
Running trial set 9997... 	2 1 	2 trials ran
Running trial set 9998... 	2 2 1 	3 trials ran
Running trial set 9999... 	2 1 	2 trials ranAverage trials ran: 2.0233
\end{lstlisting}

From the example above, our $E[X]$ was 2, so this is pretty close.

\end{document}
